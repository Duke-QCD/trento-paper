\documentclass[aps,prl,reprint,amsmath,nofootinbib]{revtex4-1}
\usepackage{hyperref}
\usepackage{graphicx}
\usepackage{placeins}
\graphicspath{{fig/}}


\begin{document}

\title{Entropy production from the generalized mean of nuclear density}

\author{J.\ Scott Moreland}
\author{Jonah E.\ Bernhard}
\author{Steffen A.\ Bass}
\affiliation{Duke University}

\date{\today}


\begin{abstract}
   We investigate entropy deposition at mid-rapidity in pp, pA and AA systems using a generalized mean of projectile and target 
densities $M_p(T_A,T_B)$ which describes a family of vector norms spanning a continuum of projectile/target saturation scenarios. 
Results are presented for three well known subcases of the generalized mean, the arithmetic, geometric and harmonic forms and 
compared against charged particle production in a wide range of collision systems. When combined with partonic density fluctuations of 
the nucleons within each nucleus, the data is well described by a generalized mean with power $p$ close to zero. We encapsulate 
these results in a new model for relativistic hydrodynamic initial conditions.
\end{abstract}

\maketitle

\section{Intoduction}

The discovery of the quark-gluon plasma (QGP) at RHIC initiated a large scale effort to quantify the transport properties of 
hot and dense nuclear matter. The success of ideal hydrodynamics in describing the collectivity observed in ultra-relativistic 
heavy-ion collisions supports the production of a fluid-like QGP with minimal specific shear viscosity close to the proposed 
K.S.S. bound $\eta/s \ge 1/4\pi$ \cite{KSS, Bjorken:1982qr}. 

Extractions of $\eta/s$ from experiment using viscous hydrodynamics describe the emergence of collective flow from spatial 
deformations of the produced fireball with a conversion efficiency proportional to the system's average specific shear viscosity. 
Consequently, model-to-data extractions of $\eta/s$ are subject to the accuracy of the simulations used to calculate entropy and 
energy deposition in the initial state.

\section{The Model}

We consider an ultra-relativistic collision between a projectile nucleus with atomic mass number $A$ and target nucleus with atomic mass number $B$ characterized 
by collision impact parameter ${\bf b}$ in the transverse plane $\vec{r}_\perp = \vec{x} \times \vec{y}$. The density of nuclear matter in the rest frame of each 
nucleus is described by the relevant un-perturbed multi-body nuclear wavefunction $\Psi_N$, where $\Psi_N$ is often characterized by the beam-integrated 
nuclear thickness function,
\begin{equation}
 T(x,y) = \int dz\, \left|\Psi_N(x,y,z) \right|^2. 
\end{equation}

Let $T_A$ and $T_B$ denote the transverse density of nucleons determined by the collapse of wavefunctions $\Psi_A$ and $\Psi_B$ in a given nucleus-nucleus 
collision. We consider the existence of an eikonal mapping $f(T_A,T_B)$ that characterizes entropy deposition at mid-rapidity as a function of $T_A$ and $T_B$,
\begin{equation}
  \label{ansatz}
  \frac{dS({\bf r}_\perp)}{d^2r_\perp dy} \biggr \rvert_{y=0}  = f(T_A({\bf r}_\perp + {\bf b}/2),\,T_B({\bf r}_\perp - {\bf b}/2)).
\end{equation}

It's reasonable to insist that this mapping $f(x_1,x_2)$ respect several key properties, namely that:  
\begin{enumerate}
 \item{$f$ is smooth} \\ $f'$, $f''$,..., $f^{(k)}$ exist and are continuous
 \item{$f$ is monatonic} \\ if $x_1 \le x'_1$ and $x_2 \le x'_2$ then $f(x_1,x_2) \le f(x'_1,x'_2)$
 \item{$f$ is symmetric} \\ $f(x_1,x_2) = f(x_2,x_1)$
 \item{$f$ obeys sensible limits} \\ $\lim\limits_{x_1,x_2 \to 0} f(x_1,x_2) = 0$.
\end{enumerate}
Motivated by these stipulations, we consider the family of vector p-norms known as the generalized mean as a 
suitable ansatz for the mapping $f$,
\begin{equation}
 \label{generalized mean ansatz}
 f(x_1, x_2) \propto M_p(x_1,x_2) = \left( \frac{x_1^p + x_2^p}{2} \right)^{1/p}.
\end{equation}

\begin{figure}[b]
 \includegraphics{saturation}
 \caption{\label{saturation}Attentuation in the generalized mean $M_p(a, 1)$ for special forms $p=-1, 0, 1$.}
\end{figure}

\begin{figure*}[t]
    \includegraphics{multdist}
    \caption{\label{fig:multdist} Minimum bias pp, pPb and PbPb charged particle distributions for power $p=0$, fluctuation parameter $k=0.8$ and normalization factor $\kappa'_n$ indicated in the legend.}
\end{figure*}

For $p = \{-\infty,-1,0,1,\infty\}$, the generalized mean becomes minimum, harmonic, geometric, arithmetic and maximum functions respectively which can be shown
to obey the following ordering inequality, 
\begin{equation}
 \mathrm{if} ~ p<p', ~ \mathrm{then} ~ M_{p}(x_1,x_2) \le M_{p'}(x_1,x_2).
\end{equation}
In this sense, the continuous parameter $p$ characterizes the attentuation of entropy deposited at mid-rapidity by each nucleon as it pierces the target nucleus. This 
attentuation is best visualized by fixing the projectile thickness $T_A$ and varying the target thickness $T_B$ as shown in figure \ref{saturation}.

There is of course one important caveat to the suitability of a generalized mean ansatz. It is easy to see in figure (\ref{fig:multdist}) that
\begin{equation}
 \lim\limits_{x_1,x_2 \to 0} f(x_1,x_2) \ne 0 ~\mathrm{for} ~ p>0.
\end{equation}
Blithely applying an arithmetic mean for example to projectile and target densities would deposit entropy in regions of the fireball where 
there is nothing to hit. 

We thus follow a common prescription and define equation \ref{ansatz} over \emph{participant} thickness functions $\mathcal{T}_A$ and $\mathcal{T}_B$ from an underlying Glauber model of 
independent nucleon-nucleon collisions with additional source fluctuations $\gamma_i$ in the transverse density $T_p$ contributed by each participant nucleon,
\begin{equation}
 \label{thickness def}
 \mathcal{T}(x,y) = \sum\limits_{i=1}^{N_{part}} \gamma_i\, T_p(x-x_i,y-y_i).
\end{equation}

The weights $\gamma_i$ are sampled from a two parameter Gamma distribution with mean $\mu=1$ and shape parameter $k$. These weights incorporate event-by-event 
fluctuations in the density of partons sampled in each nucleon-nucleon collision. The choice of a Gamma distribution reflects a long history of fitting proton-proton 
multiplicity distributions using a Negative Binomial distribution which arises naturally from a Gamma-Poisson mixture.

For the average transverse parton density $T_p(x,y)$ we use a Gaussian,
\begin{equation}
 T_p(x,y) = \frac{1}{\sqrt{2 \pi B}} \exp\left(-\frac{x^2+y^2}{2 B}\right) 
\end{equation}
with fixed nucleon width $B=0.36 ~\mathrm{fm}^2$ although we stress that this is a rather crude description of the nucleon and consequently restrict our attention 
to properties of the collision that are insensitive to details of the nucleonic substructure.

Participant nucleons are determined in the projectile and target nuclei using the probabilistic collision criteria described in reference \cite{proton-proton}. 
The pair-wise collision probability is then sampled independently for each pair of projectile and target nucleons to determine the participants which contribute in 
equation (\ref{thickness def}).

For simplicity, we refer to this generalized mean of participant densities as the \emph{reduced thickness} of projectile nucleus $A$ and target nucleus $B$,
\begin{equation}
 \label{reduced}
 \mathcal{T}_R(p;x,y) \equiv \left( \frac{\mathcal{T}_A({\bf x}+{\bf b}/2,y)^p + \mathcal{T}_B({\bf x}-{\bf b}/2,y)^p}{2} \right)^{1/p}
\end{equation}
noting that the units of $T_R$ are also that of a nuclear thickness function. In terms of the reduced thickness function, entropy deposition at mid-rapidity is 
determined according to
\begin{equation}
 \label{entropy deposition}
 \frac{dS({\bf r}_\perp)}{d^2r_\perp dy} \biggr \rvert_{y=0} = \kappa_s(\sqrt{s})\, \mathcal{T}_R(p;{\bf r}_\perp)
\end{equation}
where $\kappa_s(\sqrt{s})$ is an undetermined energy-dependent normalization factor and $\mathcal{T}_R(x,y)$ is the reduced thickness function.

Equivalently, it's been shown that the initial entropy is a close proxy for charged particle multiplicity up to viscous entropy corrections [?]. We can 
recast equation (\ref{entropy deposition}) in terms of the charged particle production as,
\begin{equation}
 \label{particle production}
 \frac{dN_{ch}({\bf r}_\perp)}{d^2r_\perp dy} \biggr \rvert_{y=0} \approx \kappa_n(\sqrt{s})\, \mathcal{T}_R(p;{\bf r}_\perp)
\end{equation}
which allows us to make a direct comparison with experimentally measured charged particle distributions.

\section{Results}

We now use the aforementioned procedure to generate $10^6$ minimum bias events satisfying $N_{part} > 0$ for proton-proton, proton-lead and lead-lead nuclei using 
realistic lead nucleon configurations incorporating short- and long-range correlations \cite{nucleon-correlations}. 

The charged particle distributions are calculated by rescaling equation (\ref{particle production}) using a modified normalization factor $\kappa'_n$ to match the average charged particle 
multiplicity in the pseudo-rapidity interval $|\eta| < 1$,
\begin{equation}
 N_{ch}(|\eta|<1) = \kappa'_n(\sqrt{s})\, \mathcal{T}_R(p;{\bf r}_\perp).
\end{equation}

In figure \ref{fig:multdist} we plot the predicted charged particle distributions for pp, pPb and PbPb systems using a geometric mean $p=0$ with fluctuation parameter $k=0.8$ tuned 
to fit the p-p multiplicity distribution. 

The normalization factor $\kappa'_n$ indicated on the figure is tuned to match the average charged particle multiplicity in each experiment. The observed
$\mathcal{O}(20\%)$ discrepancies in $\kappa'_n$ are consistent with variations in the collision energy and kinematic cuts.

In figure (\ref{fig:eccen}) we plot the Pb-Pb ellipticity and triangularity $\epsilon_2$ and $\epsilon_3$ (top and middle panels) calculated according to,
\begin{equation}
 \varepsilon_n e^{i n\phi} = -\frac{\int d^2r\, r^n e^{i n \phi} dS/dy(\vec{r})}{\int d^2r\, r^n dS/dy(\vec{r})}.
\end{equation}
along with the eccentricity ratio $\sqrt{\langle \varepsilon_2^2 \rangle}/\sqrt{\langle \varepsilon_3^2 \rangle}^{0.6}$ (bottom panel). The shaded band in the bottom 
panel indicates the values of this ratio allowed by LHC flow data and hydrodynamic calculations as determined by the authors in [?].

\begin{figure}[t]
 \includegraphics{./fig/eccentricity.pdf}
 % eccentricity.pdf: 245x227 pixel, 72dpi, 8.64x8.01 cm, bb=0 0 245 227
  \caption{\label{fig:eccen} Eccentricity harmonics $\varepsilon_2$ (top) and $\varepsilon_3$ (middle) calculated using an entropy weight for powers $p=-1,0,1$. 
  In the bottom panel the ratio $\mathcal{R} = \sqrt{\langle \varepsilon_2^2 \rangle}/\sqrt{\langle \varepsilon_3^2 \rangle}^{0.6}$ is shown with the experimentally allowed values calculated in ref. \cite{constraining-ic}}
\end{figure}

\section{Comparison to wounded nucleon and binary collision mixture}

It's interesting to compare equation (\ref{entropy deposition}) against the commonly used wounded nucleon and binary collision parameterization where the 
transverse entropy (or energy) density is set proportional to a linear combination of the local density of participant and pair-wise collisions,
\begin{equation}
 \label{glb ansatz}
 \frac{dS({\bf r}_\perp)}{d^2r_\perp dy} \biggr \rvert_{y=0}  \propto \frac{1-\alpha}{2}\mbox{WN}({\bf r_\perp}) + \alpha\, \mbox{BC}({\bf r_\perp}). 
\end{equation}

In practice, wounded nucleon and binary collision deposition is implemented according to variety of different prescriptions. We use a two-component model
with Gamma fluctuations $\gamma_i$ of the source nucelons given by,
\begin{equation}
 \label{glb algorithm}
 \frac{dS({\bf r}_\perp)}{d^2r_\perp dy} \biggr \rvert_{y=0}  \propto \sum\limits_{i=1}^{N_{part}} \gamma_i \left(\frac{1-\alpha}{2} + \alpha \frac{N_{coll,i}}{2} \right ) T_p(\vec{r}-\vec{r}_i)
\end{equation}
where $N_{coll,i}$ denotes the number of collisions suffered by the $i$-th participant. The $N_{coll}$ binary collision term in equation (\ref{glb ansatz}) scales with the product of nuclear thickness 
functions $N_{coll} \propto T_A T_B$ and predicts a rapid increase in charged particle production in systems for which the density of nuclear overlap is large.  

In contrast with the two-component model, the proposed generalized mean ansatz exhibits no such quadratic scaling. Specifically, for an idealized system in which the density of nuclear overlap is perfectly symmetric, 
i.e. $\mathcal{T}_A = \mathcal{T}_B$, the generalized mean and two-component models behave quite differently:
\begin{equation}
   \label{symmetric scaling}
   \frac{dS}{dy} \propto
  \begin{cases}
   \mathcal{T} + \mathcal{C} \mathcal{T}^2 & \text{two-component,}\\
   \mathcal{T} & \text{generalized mean}
  \end{cases}
\end{equation}
where the coefficient $\mathcal{C}$ scales with the binary collision fraction $\alpha$.

An interesting consequence of this non-linearity becomes manifest in central U-U collisions. The Uranium nucleus is highly deformed and prolonged along a preferred
axis. This geometry allows for two distinct configurations of Uranium collisions in which the density of nuclear matter is fully overlapping - tip-to-tip collisions
where the Uranium symmetry axis is alligned with the beam axis and side-on-side collisions where the Uranium symmetry axis is perpendicular to the beam axis.

Binary collision scaling predicts that tip-to-tip collisions generate more charged particles than side-on-side collisions and hence should be found at slightly
higher centralities. It was pointed out that this multiplicity ordering should imprint itself on the flow harmonics of central UU collisions due to the large 
ellipticity expected in side-on-side collisions and small ellipticity expected in body-on-body collisions. This predicted structure was dubbed the ``knee''.

\begin{figure}
 \centering
 \includegraphics{./fig/uranium.pdf}
 % uranium.pdf: 245x151 pixel, 72dpi, 8.64x5.33 cm, bb=0 0 245 151
 \caption{\label{fig:knee} Scaled TRENTO ($p=0$, $k=0.8$, $B=0.36~\mathrm{fm}^2$) and fluctuated Glauber ($\alpha=14$, $k=1.2$, $B=0.36~\mathrm{fm}^2$) ellipticities
 compared to STAR elliptic flow data.}
\end{figure}

In figure (\ref{fig:knee}) we compare the predictions of the fluctuated two-component (eq. \ref{glb algorithm}) and generalized mean prescriptions
(eq. \ref{entropy deposition}) using $10^6$ minimum bias events in the impact parameter range $0<b<3.5 ~\mathrm{fm}$ with a $15\%$ cut to avoid 
contamination from events $b>3.5 ~\mathrm{fm}$.

The fluctuated two component model shows clear evidence of a knee structure in the $0-0.5\%$ centrality bin which is inconsistent with the STAR data. These findings agree 
with those previously reported in a comprehensive flow analysis by A. Goldschmidt, Zhi Qiu and U. Heinz, in preparation and private communication. The authors in [?] 
found that a hot-spot model with sharply peaked entropy density fluctuations described by concentrated semi-hard cross-section $\sigma \approx 2 ~\mathrm{mb}$ is sufficient to wash 
out the structure of the knee in UU collisions, but it remains to be seen if such sharply peaked hot-spots fit into a coherent description of charged particle production in 
small systems.

In contrast to the two-component model, the generalized mean correctly reproduces the shape of the ultra-central elliptic flow measured by STAR. This suggests that non-linearities
in the mapping discussed in equation (\ref{symmetric scaling}) are small and it supports attentuation as the mechanism which describes the centrality dependence of charged particle production
in AA collisions.


\section{Summary}

Initial conditions currently comprise the largest source of uncertainty in hydrodynamic simulations of the QGP fireball [refs]. Ideally, one seeks a first principles calculation 
of the initial state (i.e. see refs), but these derivations are often irrelevant to the problem of interest making effective models an attractive alternative.

We propose the family of vector p-norms known as the generalized mean as an effective model for entropy production in ultra-relativistic nucleus-nucleus collisions and and demonstrate the model's
ability to replicate pp, pPb and PbPb multiplicity distributions. In figure \ref{fig:eccen} we show that the predicted ellipticity of the produced fireball is highly sensitive to the power of 
the p-norm while the fluctuation driven triangularity is less sensitive. The optimal ratio of these harmonics which was extracted from flow data in reference [?] prefers a value of p close to 0 
corresponding to the geometric mean. We note that geometric mean scaling of the initial entropy was also reported by an AdS-CFT calculation in reference [?]. 


We also compare the generalized mean ansatz against a fluctuated two-component model in Uranium-Uranium collisions where one expects the eccentricity harmonics to behave quite differently. The generalized mean
correctly predicts the shape of the elliptic flow in ultra-central UU collisions while the two-component model exhibits a well known knee structure which is inconsistent with the STAR data.


\bibliography{sources}

\end{document}
