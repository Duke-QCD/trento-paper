\documentclass[aps,prl,reprint,amsmath,nofootinbib]{revtex4-1}

\usepackage{hyperref}
\usepackage{graphicx}
\graphicspath{{fig/}}

\usepackage{mdwlist}
\renewcommand\labelitemi{\raisebox{.3ex}{\tiny$\bullet$}}

\newcommand{\trento}{T\raisebox{-.5ex}{R}ENTo}
\newcommand{\nch}{N_\text{ch}}
\newcommand{\needcite}{\textbf{[???]}}

\begin{document}

\title{A unified, effective model for particle production in high-energy nuclear collisions}

\author{J.\ Scott Moreland}
\author{Jonah E.\ Bernhard}
\author{Steffen A.\ Bass}
\affiliation{Duke University}

\date{\today}


\begin{abstract}
  The new \trento\ initial condition model is assembled using an effective description of entropy production
  in high-energy nuclear collisions.  Self-consistent pp, pPb, and PbPb multiplicity distributions and PbPb
  eccentricity harmonics are calculated and compare favorably to experiment.  A possible explanation of recent
  uranium-uranium results is proposed.
\end{abstract}


\maketitle

\section{Introduction}

Viscous relativistic fluid dynamics provides a stable, well-tested model for the time-evolution and collective
behavior in high-energy nuclear collisions.  However, a realistic Monte Carlo (MC) model for the initial state
of the medium is essential for describing many event-by-event observables.  These initial conditions remain
one of the most poorly-constrained components of modern computational models.

Initial condition models create profiles of energy or entropy at some early time which are then evolved by
fluid dynamics.  Perhaps the simplest prescription is the so-called wounded nucleon model, which determines
participating nucleons via optical overlap and deposits a blob of energy for each participant.  The MC-Glauber
model generalizes wounded nucleons by additionally depositing entropy proportional to the number of binary
nucleon-nucleon collisions.  Despite its simplicity, the Glauber model has qualitatively fit a variety of
experimental measurements.

More recently, the IP-Glasma model applied color-glass condensate (CGC) effective field theory to create a
dynamical description of the pre-fluid stage of the collision.  IP-Glasma has quantitatively fit many
experimental data.  However, it is theoretically and computationally complex and its applicability to smaller
collision systems (e.g.~proton-proton) is uncertain.

In this letter, we present a new unified initial condition model for high-energy collisions ranging from
proton-proton to nucleus-nucleus.


\section{Model}

The model is constructed in a similar fashion to the two-component Glauber model, but using a different ansatz
for entropy production.

Suppose a pair of projectiles labeled $A$, $B$ collide along beam axis $z$.  Each projectile is represented by
the beam-integrated density of its nuclear matter, typically called thickness:
\begin{equation}
  T_{A,B}(x, y) = \int dz \, \rho_{A,B}(x, y, z).
\end{equation}
The construction of the thickness functions will be addressed in the following subsections; first, we
postulate the following:
\begin{enumerate}
  \item The eikonal approximation is valid:  entropy is produced if $T_A$ and $T_B$ eikonally overlap.
  \item There exists a scalar field $f(T_A, T_B)$ which converts projectile thicknesses into entropy
    deposition.
\end{enumerate}
The function $f$ is proportional to the entropy created at mid-rapidity and at the hydrodynamic thermalization
time:
\begin{equation}
  f \propto dS/dy \, |_{\tau = \tau_0}.
\end{equation}
It should provide an \emph{effective} description of early collision dynamics:  it need not arise from a
first-principles calculation, but must obey basic physical restrictions.

With this in mind, we introduce for $f$ the \emph{reduced thickness}
\begin{equation}
  f = T_R(p; T_A, T_B) \equiv \biggl( \frac{T_A^p + T_B^p}{2} \biggr)^{1/p},
  \label{eq:tr}
\end{equation}
so named because it takes two thicknesses $T_A$, $T_B$ and ``reduces'' them to a third thickness, similar to a
reduced mass.  The dimensionless parameter $p$ may take any real value in $(-\infty, \infty)$ and is to be
determined by experiment.  This functional form---known as the generalized mean---simplifies to the arithmetic,
geometric, and harmonic mean for certain values of $p$, i.e.
\begin{equation}
  T_R(p; T_A, T_B) =
  \begin{cases}
    \dfrac{T_A + T_B}{2} & p = 1 \text{ (arithmetic)}, \\[2ex]
    \sqrt{T_A T_B} & p = 0 \text{ (geometric)}, \\[2ex]
    \dfrac{2 T_A T_B}{T_A + T_B} & p = -1 \text{ (harmonic)}. \\
  \end{cases}
\end{equation}
More generally, $p$ quantifies the amount of attenuation in asymmetric ($T_A \neq T_B$) regions of the
collision.  By attenuation, we mean \textbf{WHAT?}, as shown in FIG.~\ref{fig:saturation}.  As $p$
\emph{decreases}, the degree of attenuation \emph{increases}.  This significantly impacts the behavior of the
model, for instance $p=1$ is precisely a wounded nucleon model, while $-1 \lesssim p \lesssim 0$ mimics CGC
saturation.

The reduced thickness possesses several other key properties.  It is
\begin{itemize*}
  \item continuous;
  \item monotonically increasing:  \\
    if $T_A \leq T_A'$ and $T_B \leq T_B'$, then $T_R \leq T_R'$;
  \item symmetric in the projectile thicknesses: \\
    $T_R(p; T_A, T_B) = T_R(p; T_B, T_A)$, and hence
  \item independent of $p$ when the thicknesses are equal: \\
    $T_R(p; T, T) = T$.
  \item The reduced thickness vanishes when both $T_A$ and $T_B$ vanish: $T_R(p; 0, 0) = 0$.  In fact, for $p
    \leq 0$, $T_R$ vanishes if \emph{either} $T_A$ or $T_B$ do.  On the other hand, positive values of $p$
    introduce small violations of the eikonal entropy production postulate, similar to a traditional wounded
    nucleon model.
\end{itemize*}

Finally, the reduced thickness provides the basis of the model name:
\trento, for Thickness-Reduced Event-by-event Nuclear Topology.

We now detail the construction of the thickness functions $T_{A,B}(x, y)$, which combined with the definition
of the reduced thickness completes the specification of the model.  The procedure is constructed from the
ground up to handle a variety of collision systems; we begin with the simplest case.

\begin{figure}[t]
  \includegraphics{saturation}
  \caption{
    \label{fig:saturation}
    Attenuation of the reduced thickness.  Thickness $T_B$ is fixed to one (in arbitrary units) while $T_A$ is
    varied.  The corresponding $T_R$ are shown for $p = 1$, 0, $-1$ (blue, green, and red).
  }
\end{figure}

\subsection{Proton-proton collisions}

Consider a collision of two protons $A$, $B$ with impact parameter $b$ along the $x$-direction.
Denote their nuclear densities by
\begin{equation}
  \rho_{A,B} = \rho_\text{proton}(x \pm b/2, y, z),
\end{equation}
and let us assume that the integral $\int dz \, \rho_\text{proton}$ either has a closed form or may be
evaluated numerically, so that the proton thickness functions can be calculated.

We now randomly decide whether the protons collide according to the probability \cite{proton-proton}
\begin{equation}
  P_\text{coll} = 1 - \exp\biggl[ -\sigma_{gg} \int dx \, dy \int dz \, \rho_A \int dz \, \rho_B \biggr],
  \label{eq:pcoll}
\end{equation}
where the integral in the exponential is the overlap integral of the proton thickness functions and
$\sigma_{gg}$ is a cross-section which is set so that the total proton-proton cross-section equals the
experimental inelastic nucleon nucleon cross-section $\sigma_{NN}$.

Assuming the protons collide, each is assigned a \emph{fluctuated} thickness
\begin{equation}
  T_{A,B}(x, y) = \gamma_{A,B} \int dz \, \rho_{A,B}(x, y),
\end{equation}
where $\gamma_{A,B}$ are independent random numbers sampled from a gamma distribution with unit mean,
\begin{equation}
  P(\gamma; k) \propto \gamma^{k-1} e^{-\gamma/k},
\end{equation}
with shape parameter $k > 0$ to be fixed by experiment.  The gamma distribution is chosen for its
flexibility---it is exponential for $k = 1$ and becomes Gaussian for large $k$---and because it is the
continuous analog of the negative binomial distribution, which has historically been used to fit proton-proton
multiplicity fluctuations.

The reduced thickness is then calculated from the projectile thickness functions; this furnishes the initial
transverse entropy profile up to an overall normalization factor, $dS/dy \propto T_R(p; T_A, T_B)$.

\subsection{Larger systems}

Composite collision systems such as proton-nucleus and nucleus-nucleus are essentially treated as
superpositions of proton-proton collisions.  A set of nucleon positions is chosen for each
projectile $A$, $B$, then the collision probability \eqref{eq:pcoll} is sampled for each pairwise interaction.
Those nucleons that collide with at least one partner are labeled ``participants'' and the rest are discarded.
The fluctuated thickness function for projectile $A$ is then
\begin{equation}
  T_A = \sum_i \gamma_i \int dz \, \rho_\text{proton}(x_i, y_i, z_i),
\end{equation}
$\gamma_i$ and $(x_i, y_i, z_i)$ are the random fluctuation factor and position, respectively, of participant
$i$ in projectile $A$.  $T_B$ is constructed using the same definition over its participants.

\textbf{SOMETHING HERE?}


\begin{figure*}[t]
  \includegraphics{multdist}
  \caption{
    \label{fig:multdist}
    Minimum bias pp, pPb and PbPb charged particle distributions for power $p=0$, fluctuation parameter
    $k=0.8$ and normalization factor $\kappa'_n$ indicated in the legend.
  }
\end{figure*}


\section{Results}

\trento\ simulations depend on several undetermined parameters---to rigorously constrain these parameters
would require a systematic model-to-data comparison which is beyond the scope of this work.  Therefore, the
following results do not necessarily represent the best-fit of the model to data.

\subsection{Multiplicity distributions}

Charged-particle multiplicity $\nch$ is to an excellent approximation proportional to the total initial
entropy \needcite, and hence proportional to the integrated reduced thickness:
\begin{equation}
  \nch \propto \int dx \, dy \, T_R.
\end{equation}
To compare with experiment, we generate a large ensemble of minimum-bias events, integrate their $T_R$
profiles, and multiply by a normalization constant to rescale the distribution to the experimental mean.

The left panel of FIG.~\ref{fig:multdist} shows the $\nch$ distribution from $10^6$ proton-proton simulations
using reduced thickness parameter $p = 0$ (geometric mean), gamma fluctuation parameter $k = 0.8$, and
Gaussian beam-integrated proton density
\begin{equation}
  \int dz \, \rho_\text{proton} = \frac{1}{2\pi B} \exp\biggr( -\frac{x^2 + y^2}{2B} \biggr)
\end{equation}
with effective area $B = (0.6\;\text{fm})^2$.  Results compare favorably with the experimental distribution
measured by ALICE \needcite.

Proton-lead and lead-lead distributions are presented in the middle and right panels of
FIG.~\ref{fig:multdist}.  We use lead nucleus configurations including nucleon-nucleon correlations from
\cite{nucleon-correlations}; model parameters are otherwise identical to proton-proton.

The model is able to reproduce the shapes of all three distributions, although the normalization factors
(annotated in the figure) only agree within $\sim$20\%.  This could be a shortcoming of the model or may be
attributable to differences in beam energy and kinematic cuts.

\subsection{Eccentricity harmonics}

\begin{figure}[b]
  \includegraphics{eccentricity}
  \caption{
    \label{fig:eccen}
    Eccentricity harmonics $\varepsilon_2$ (top) and $\varepsilon_3$ (middle) as a function of centrality for
    reduced thickness parameters $p = -1$, 0, 1.  The bottom panel shows the ratio of the rms eccentricities
    $\sqrt{\langle \varepsilon_2^2 \rangle}/\sqrt{\langle \varepsilon_3^2 \rangle}^{\,0.6}$ against the
    experimentally allowed region (grey band) from \cite{constraining-ic}.
  }
\end{figure}

Eccentricity harmonics $\varepsilon_n$ are calculated for lead-lead collisions using the definition
\begin{equation}
  \varepsilon_n e^{i n\phi} = -\frac{\int dx \, dy\, r^n e^{i n \phi} \, T_R}{\int dx \, dy \, r^n \, T_R}.
\end{equation}
Figure~\ref{fig:eccen} shows ellipticity $\varepsilon_2$ and triangularity $\varepsilon_3$ as a function of
centrality for reduced thickness $p = 1$,~0,~$-1$.  There is a clear trend of increasing eccentricity
(particularly $\varepsilon_2$) with decreasing $p$.  This may be understood by the attenuation property of the
reduced thickness:  as $p$ decreases, asymmetric regions of the collision produce less entropy, which
accentuates the elliptical overlap shape in noncentral collisions and enhances eccentricity.

In addition, we perform the test described in \cite{constraining-ic}.  The authors use experimental flow data
and hydrodynamics to construct an experimentally allowed band of the ratio of root-mean-square eccentricities
$\sqrt{\langle \varepsilon_2^2 \rangle}/\sqrt{\langle \varepsilon_3^2 \rangle}^{\,0.6}$ as a function of
centrality.  They check a number of common initial condition models and find that only IP-Glasma consistently
falls within the allowed region.  The bottom panel of FIG.~\ref{fig:eccen} shows that the geometric mean
($p = 0$) yields excellent agreement with experiment.

\subsection{Ultracentral uranium-uranium}

\begin{figure}[t]
  \centering
  \includegraphics{uranium}
  \caption{
    \label{fig:uranium}
    Ellipticity $\varepsilon_2$ as a function of charged-particle multiplicity for ultracentral
    uranium-uranium collisions.
  }
\end{figure}

Uranium nuclei have a deformed prolate spheroid shape, thus a uranium-uranium collision may achieve maximal
overlap via two distinct orientations:  ``tip-tip'', in which the long axes of the spheroids are aligned with
the beam axis and the overlap area is circular; or ``side-side'', where the long axes are perpendicular to the
beam axis and the overlap area is elliptical.

In the two-component Glauber model, tip-tip collisions have the same number wounded nucleons but more binary
collisions than side-side, so tip-tip would have larger multiplicity.  Therefore, the Glauber model predicts a
rapid decrease in elliptic flow $v_2$ as a function of multiplicity in the most central uranium-uranium
collisions.  This structure is known as the ``knee'' \needcite.

The STAR experiment at RHIC recently performed uranium-uranium collisions and found no evidence of a knee
\cite{UU-STAR}.  It has been proposed that fluctuations could wash out the knee \needcite, but a recent
comprehensive flow analysis showed that the knee would still be visible \cite{osu}.

The reduced thickness ansatz \eqref{eq:tr} does not have binary collision scaling and hence produces roughly
constant ellipticity in ultracentral uranium-uranium, as shown in FIG.~\ref{fig:uranium}.  Short of conducting
a full hydrodynamic analysis, these results appear to be more consistent with STAR data than Glauber, and are
similar to IP-Glasma \needcite.


\section{Conclusion}

We introduce \trento, a new unified initial condition model for high-energy nuclear collisions based on
eikonal entropy creation via the \emph{reduced thickness}.  The model is able to simultaneously fit
proton-proton, proton-lead, and lead-lead multiplicity distributions along with lead-lead eccentricity harmonics,
and provides a possible resolution to the ``knee'' problem in ultracentral uranium-uranium collisions.
It is flexible by design and able to mimic the behavior of other initial condition models.

This is an intentionally \emph{effective} model.  A first-principles derivation would be ideal, but such
models are usually complex and require significant computation time.  So in practical terms, effective models
are often preferable for empirical modeling.

We plan to publicly release \trento\ in the coming weeks, and the community is encouraged to use, test, and
contribute to the source code.



\bibliography{sources}


\end{document}
