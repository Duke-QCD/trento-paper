\documentclass[aps,prl,reprint,amsmath,nofootinbib]{revtex4-1}
\usepackage{hyperref}
\usepackage{graphicx}
\usepackage{placeins}
\graphicspath{{fig/}}


\begin{document}

\title{Entropy production from the generalized mean of nuclear density}

\author{J.\ Scott Moreland}
\author{Jonah E.\ Bernhard}
\author{Steffen A.\ Bass}
\affiliation{Duke University}

\date{\today}


\begin{abstract}
   We investigate entropy deposition at mid-rapidity in pp, pA and AA systems using a generalized mean of projectile and target 
densities $M_p(T_A,T_B)$ which describes a family of vector norms spanning a continuum of projectile/target saturation scenarios. 
Results are presented for three well known subcases of the generalized mean, the arithmetic, geometric and harmonic forms and 
compared against charged particle production in a wide range of collision systems. When combined with partonic density fluctuations of 
the nucleons within each nucleus, the data is well described by a generalized mean with power $p$ close to zero. We encapsulate 
these results in a new model for relativistic hydrodynamic initial conditions.
\end{abstract}

\maketitle

\section{Intoduction}

The discovery of the quark-gluon plasma (QGP) at RHIC initiated a large scale effort to quantify the transport properties of 
hot and dense nuclear matter. The success of ideal hydrodynamics in describing the collectivity observed in ultra-relativistic 
heavy-ion collisions supports the production of a fluid-like QGP with minimal specific shear viscosity close to the proposed 
K.S.S. bound $\eta/s \ge 1/4\pi$ \cite{KSS}. 

Extractions of $\eta/s$ from experiment using viscous hydrodynamics describe the emergence of collective flow from spatial 
deformations of the produced fireball with a conversion efficiency proportional to the system's average specific shear viscosity [?]. 
Consequently, model-to-data predictions of $\eta/s$ are beholden to the veractity of theoretical models used to calculate entropy and 
energy deposition in the initial state.

\section{The Model}

We consider an ultra-relativistic collision between a projectile nucleus with atomic mass number $A$ and target nucleus with atomic mass number $B$ characterized 
by collision impact parameter ${\bf b}$ in the transverse plane $\vec{r}_\perp = \vec{x} \times \vec{y}$. The density of nuclear matter in the rest frame of each 
nucleus is described by the relevant un-perturbed multi-body nuclear wavefunction $\Psi_N$, where $\Psi_N$ is often characterized by the beam-integrated 
nuclear thickness function,
\begin{equation}
 T(x,y) = \int dz\, \left|\Psi_N(x,y,z) \right|^2. 
\end{equation}

Let $T_A$ and $T_B$ denote the transverse density of nucleons determined by the collapse of wavefunctions $\Psi_A$ and $\Psi_B$ in a given nucleus-nucleus 
collision. We consider the existence of an eikonal mapping $f(T_A,T_B)$ that characterizes entropy deposition at mid-rapidity as a function of $T_A$ and $T_B$,
\begin{equation}
  \label{ansatz}
  \frac{dS({\bf r}_\perp)}{d^2r_\perp dy} \biggr \rvert_{y=0}  = f(T_A({\bf r}_\perp + {\bf b}/2),\,T_B({\bf r}_\perp - {\bf b}/2)).
\end{equation}

It's reasonable to insist that $f$ respect several key properties, namely that $f$ is smooth, monatonic and symmetric in the variables $T_A$ and 
$T_B$ and that it saturates in the limit $T_A \ll T_B$ and $T_B \ll T_A$. Motivated by these stipulations, we consider the family of vector p-norms 
known as the generalized mean as a suitable ansatz for the mapping $f$,
\begin{equation}
 \label{generalized mean ansatz}
 f(T_A, T_B) \propto M_p(T_A,T_B) = \left( \frac{T_A^p + T_B^p}{2} \right)^{1/p}.
\end{equation}

For $p = \{-\infty,-1,0,1,\infty\}$, the generalized mean becomes minimum, harmonic, geometric, arithmetic and maximum functions respectively which can be shown
to obey the following ordering inequality, 
\begin{equation}
 M_{-\infty} \le M_{-1} \le M_{0} \le M_{1} \le M_{\infty}.
\end{equation}
In this sense, the continuous parameter $p$ characterizes the dependence of entropy deposition on the projectile-target participant density asymmetry, or equivalently 
the degree of collision saturation. This saturation on $p$ is best visualized by fixing the thickness of the projectile $T_A$ and varying the thickness of the target 
$T_B$ for different values of $p$ as shown in figure \ref{saturation}.
\begin{figure}[b]
 \includegraphics{saturation}
 \caption{\label{saturation}Saturation in the generalized mean $M_p(a, 1)$ for special forms $p=\{-1, 0, 1\}$.}
\end{figure}

For simplicity, we refer to the generalized mean of participant projectile and target thicknesses hereafter as the \emph{reduced thickness} of projectile 
nucleus $A$ and target nucleus $B$,
\begin{equation}
 \label{reduced}
 \mathcal{T}_R(p;x,y) \equiv \left( \frac{\mathcal{T}_A({\bf x}+{\bf b}/2,y)^p + \mathcal{T}_B({\bf x}-{\bf b}/2,y)^p}{2} \right)^{1/p}
\end{equation}
noting that the units of $T_R$ are also that of a nuclear thickness function. 

The participant nuclear thickness functions $\mathcal{T}_A$ and $\mathcal{T}_B$ are calculated from an underlying Glauber model of independent nucleon-nucleon 
collisions with additional fluctuations $\gamma_i$ in the transverse density $T_p$ contributed by each participant nucleon,
\begin{equation}
 \label{thickness def}
 \mathcal{T}(x,y) = \sum\limits_{i=1}^{N_{part}} \gamma_i\, T_p(x-x_i,y-y_i).
\end{equation}

The weights $\gamma_i$ are sampled from a two parameter Gamma distribution with mean $\mu=1$ and shape parameter $k$. These weights incorporate event-by-event 
fluctuations in the density of partons sampled in each nucleon-nucleon collision. The choice of a Gamma distribution reflects a long history of fitting proton-proton 
multiplicity distributions using a negative binomial distribution which arises naturally from a Gamma-Poisson mixture.

For the average transverse parton density $T_p(x,y)$ we use a Gaussian,
\begin{equation}
 T_p(x,y) = \frac{1}{\sqrt{2 \pi B}} \exp\left(-\frac{x^2+y^2}{2 B}\right) 
\end{equation}
although we stress that this is a rather crude description of the nucleon and consequently restrict our attention to properties of the collision that are 
insensitive to details of the nucleonic substructure.

Participant nucleons are determined in the projectile and target nuclei using the probabilistic collision criteria described in reference \cite{proton-proton}. 
The pair-wise collision probability is then sampled independently for each pair of projectile and target nucleons to determine the 
participant nucleons which contribute to the summation in equation (\ref{thickness def}).

We finally fix the density of deposited entropy at mid-rapidity according to,
\begin{equation}
 \label{entropy deposition}
 \frac{dS({\bf r}_\perp)}{d^2r_\perp dy} \biggr \rvert_{y=0} = \kappa_s(\sqrt{s})\, \mathcal{T}_R(p;{\bf r}_\perp)
\end{equation}
where $\kappa_s(\sqrt{s})$ is an undetermined energy-dependent normalization factor and $\mathcal{T}_R(x,y)$ is the reduced thickness function.

Equivalently, it's been shown that the initial entropy is a close proxy for charged particle multiplicity up to viscous entropy corrections [?]. We can 
recast equation (\ref{entropy deposition}) in terms of the charged particle production as,
\begin{equation}
 \label{particle production}
 \frac{dN_{ch}({\bf r}_\perp)}{d^2r_\perp dy} \biggr \rvert_{y=0} \approx \kappa_n(\sqrt{s})\, \mathcal{T}_R(p;{\bf r}_\perp)
\end{equation}
which allows us to make a direct comparison with experimentally measured charged particle distributions.

\section{Results}

We show results for proton-proton, proton-lead and lead-lead nuclei using realistic lead nucleon configurations incorporating short- and long-range 
correlations \cite{nucleon-correlations}. For each collision system we generate $10^6$ minimum bias events satisfying $N_{part} > 0$. Charged particle
production at mid-rapidity is calculated from equation (\ref{particle production}) according to, 
\begin{equation}
 N_{ch}(|\eta|<1) = \kappa'_n(\sqrt{s})\, \mathcal{T}_R(p;{\bf r}_\perp)
\end{equation}
where the normalization factor $\kappa'_n$ is scaled to match the average charged particle multiplicity in the pseudo-rapidity interval $|\eta| < 1$.

In figure (\ref{fig:eccen}) we plot the Pb-Pb ellipticity and triangularity $\epsilon_2$ and $\epsilon_3$

along with the correpsonding ratio $\sqrt{\langle \varepsilon_2^2 \rangle}/\sqrt{\langle \varepsilon_3^2 \rangle}^{0.6}$ (right pannel). The gray band, 
taken from reference [?], indicates the value of this ratio preferred by the LHC flow data.

\begin{figure}[b]
 \includegraphics{./fig/eccentricity.pdf}
 % eccentricity.pdf: 245x227 pixel, 72dpi, 8.64x8.01 cm, bb=0 0 245 227
  \caption{\label{fig:eccen} Eccentricity harmonics $\varepsilon_2$ (left) and $\varepsilon_3$ (middle) calculated using an entropy weight for powers $p=-1,0,1$ and
  the ratio = $\sqrt{\langle \varepsilon_2^2 \rangle}/\sqrt{\langle \varepsilon_3^2 \rangle}^{0.6}$ (right).}
\end{figure}

In figure \ref{fig:multdist} we illustrate charged particle predictions against the ALICE data using a geometric mean $p=0$ with a fluctuation parameter $k=0.8$ tuned 
to fit the p-p multiplicity distribution and a squared nucleon width fixed at the rather standard value $B=0.36$ $\mbox{fm}^2$. 

\begin{figure*}
    \includegraphics{multdist}
    \caption{\label{fig:multdist} Minimum bias pp, pPb and PbPb charged particle distributions for power $p=0$, fluctuation parameter $k=0.8$, squared nucleon width 
    $B=0.36~fm^2$ and normalization factor $\kappa'_n$ indicated in the legend.}
\end{figure*}

The normalization factor $\kappa'_n$ indicated on the figure is tuned to match the average charged particle multiplicity in each experiment. The observed
$\mathcal{O}(20\%)$ discrepancies in $\kappa'_n$ are consistent with variations in the collision energy and kinematic cuts.

\section{Comparison to wounded nucleon and binary collision mixture}

It's interesting to compare equation (\ref{entropy deposition}) against the commonly used wounded nucleon and binary collision parameterization where the 
transverse entropy (or energy) density is set proportional to a linear combination of the local density of participant and pair-wise collisions,
\begin{equation}
 \label{glb ansatz}
 \frac{dS({\bf r}_\perp)}{d^2r_\perp dy} \biggr \rvert_{y=0}  \propto \frac{1-\alpha}{2}\mbox{WN}({\bf r_\perp}) + \alpha\, \mbox{BC}({\bf r_\perp}). 
\end{equation}

In practice, wounded nucleon and binary collision deposition is implemented according to variety of different prescriptions. We use a two-component model
with Gamma fluctuations $\gamma_i$ of the source nucelons given by,
\begin{equation}
 \label{glb algorithm}
 \frac{dS({\bf r}_\perp)}{d^2r_\perp dy} \biggr \rvert_{y=0}  \propto \sum\limits_{i=1}^{N_{part}} \gamma_i \left(\frac{1-\alpha}{2} + \alpha \frac{N_{coll,i}}{2} \right ) T_p(\vec{r}-\vec{r}_i)
\end{equation}
where $N_{coll,i}$ denotes the number of collisions suffered by the $i$-th participant. The $N_{coll}$ binary collision term in equation (\ref{glb ansatz}) scales with the product of nuclear thickness 
functions $N_{coll} \propto T_A T_B$ and predicts a rapid increase in charged particle production in systems for which the density of nuclear overlap is large.  

In contrast with the two-component model, the proposed generalized mean ansatz exhibits no such quadratic scaling. Specifically, for an idealized system in which the density of nuclear overlap is perfectly symmetric, 
i.e. $\mathcal{T}_A = \mathcal{T}_B$, the generalized mean and two-component models behave quite differently:
\begin{equation}
   \label{symmetric scaling}
   \frac{dS}{dy} \propto
  \begin{cases}
   \mathcal{T} + \mathcal{C} \mathcal{T}^2 & \text{two-component,}\\
   \mathcal{T} & \text{generalized mean}
  \end{cases}
\end{equation}
where the coefficient $\mathcal{C}$ scales with the binary collision fraction $\alpha$.

An interesting consequence of this non-linearity becomes manifest in central U-U collisions. The Uranium nucleus is highly deformed and prolonged along a preferred
axis. This geometry allows for two distinct configurations of Uranium collisions in which the density of nuclear matter is fully overlapping - tip-to-tip collisions
where the Uranium symmetry axis is alligned with the beam axis and side-on-side collisions where the Uranium symmetry axis is perpendicular to the beam axis.

Binary collision scaling predicts that tip-to-tip collisions generate more charged particles than side-on-side collisions and hence should be found at slightly
higher centralities. It was pointed out that this multiplicity ordering should imprint itself on the flow harmonics of central UU collisions due to the large 
ellipticity expected in side-on-side collisions and small ellipticity expected in body-on-body collisions. This predicted structure was dubbed the ``knee''.

\begin{figure}[h]
 \centering
 \includegraphics{./fig/uranium.pdf}
 % uranium.pdf: 245x151 pixel, 72dpi, 8.64x5.33 cm, bb=0 0 245 151
 \caption{\label{fig:knee} Scaled TRENTO ($p=0$, $k=0.8$, $B=0.36~\mathrm{fm}^2$) and fluctuated Glauber ($\alpha=14$, $k=1.2$, $B=0.36~\mathrm{fm}^2$) ellipticities
 compared to STAR elliptic flow data.}
\end{figure}

In figure (\ref{fig:knee}) we compare the predictions of fluctuated two-component (eq. \ref{glb algorithm}) and generalized mean prescriptions
(eq. \ref{entropy deposition}) using $10^6$ minimum bias events in the impact parameter range $0<b<3.5 ~\mathrm{fm}$ with a $15\%$ cut to avoid 
contamination from events $b>3.5 ~\mathrm{fm}$.

We see that the generalized mean correctly reproduces the elliptic flow data in ultra-central UU collisions while the two-component fit is poor. The distinct knee structure seen
in the fluctuated two-component model is not observed by STAR. 


\section{Conclusion}

We present a new model for relativistic heavy-ion collisions that uses a generalized mean to interpolate between different degrees of collision attentuation. Recent results from STAR UU collisions indicate that attentuation is consistent 

An interesting consequence of this non-linearity becomes manifest in central U-U collisions. The Uranium nucleus is highly deformed and prolonged along a preferred
axis. This geometry allows for two distinct configurations of Uranium collisions in which the density of nuclear matter is fully overlapping - tip-to-tip collisions
where the Uranium symmetry axis is alligned with the beam axis and side-on-side collisions where the Uranium symmetry axis is perpendicular to the beam axis.

Binary collision scaling predicts that tip-to-tip collisions generate more charged particles than side-on-side collisions and hence should be found at slightly
higher centralities. It was pointed out that this multiplicity ordering should imprint itself on the flow harmonics of central UU collisions due to the large 

\bibliography{sources}

\end{document}
