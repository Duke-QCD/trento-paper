\documentclass[aps,prl,reprint,amsmath,nofootinbib]{revtex4-1}

\usepackage{hyperref}
\usepackage{graphicx}
\graphicspath{{fig/}}

\usepackage{mdwlist}
\renewcommand\labelitemi{\raisebox{.3ex}{\tiny$\bullet$}}

\newcommand{\trento}{T\raisebox{-.5ex}{R}ENTo}
\newcommand{\nch}{N_\text{ch}}
\newcommand{\needcite}{\textbf{[???]}}

\begin{document}

\title{Entropy production from the generalized mean of nuclear density}

\author{J.\ Scott Moreland}
\author{Jonah E.\ Bernhard}
\author{Steffen A.\ Bass}
\affiliation{Duke University}

\date{\today}


\begin{abstract}
  We investigate entropy deposition at mid-rapidity in pp, pA and AA systems using a generalized mean of
  projectile and target densities $M_p(T_A,T_B)$ which describes a family of vector norms spanning a continuum
  of projectile/target saturation scenarios.  Results are presented for three well known subcases of the
  generalized mean, the arithmetic, geometric and harmonic forms and compared against charged particle
  production in a wide range of collision systems. When combined with partonic density fluctuations of the
  nucleons within each nucleus, the data is well described by a generalized mean with power $p$ close to zero.
  We encapsulate these results in a new model for relativistic hydrodynamic initial conditions.
\end{abstract}


\maketitle

\section{Introduction}

Viscous relativistic fluid dynamics provides a stable, well-tested model for the time-evolution and collective
behavior in high-energy nuclear collisions.  However, a realistic Monte Carlo (MC) model for the initial state
of the medium is essential for describing many event-by-event observables.  These initial conditions remain
one of the most poorly-constrained components of modern computational models.

Initial condition models create profiles of energy or entropy at some early time which are then evolved by
fluid dynamics.  Perhaps the simplest prescription is the so-called wounded nucleon model, which determines
participating nucleons via optical overlap and deposits a blob of energy for each participant.  The MC-Glauber
model generalizes wounded nucleons by additionally depositing entropy proportional to the number of binary
nucleon-nucleon collisions.  Despite its simplicity, the Glauber model has qualitatively fit a variety of
experimental measurements.

More recently, the IP-Glasma model applied color-glass condensate (CGC) effective field theory to create a
dynamical description of the pre-fluid stage of the collision.  IP-Glasma has quantitatively fit many
experimental data.  However, it is theoretically and computationally complex and its applicability to smaller
collision systems (e.g.~proton-proton) is uncertain.

In this letter, we present a new unified initial condition model for high-energy collisions ranging from
proton-proton to nucleus-nucleus.


\section{Model}

The model is constructed in a similar fashion to the two-component Glauber model, but using a different ansatz
for entropy production.

Suppose a pair of projectiles labeled $A$, $B$ collide along beam axis $z$.  Each projectile is represented by
the beam-integrated density of its nuclear matter, typically called thickness:
\begin{equation}
  T_{A,B}(x, y) = \int dz \, \rho_{A,B}(x, y, z).
\end{equation}
The construction of the thickness functions will be addressed in the following subsections; first, we
postulate the following:
\begin{enumerate}
  \item The eikonal approximation is valid:  entropy is produced if $T_A$ and $T_B$ eikonally overlap.
  \item There exists a scalar field $f(T_A, T_B)$ which converts projectile thicknesses into entropy
    deposition.
\end{enumerate}
The function $f$ is proportional to the entropy created at mid-rapidity and at the hydrodynamic thermalization
time:
\begin{equation}
  f \propto dS/dy \, |_{\tau = \tau_0}.
\end{equation}
It should provide an \emph{effective} description of early collision dynamics:  it need not arise from a
first-principles calculation, but must obey basic physical restrictions.

With this in mind, we introduce for $f$ the \emph{reduced thickness}
\begin{equation}
  f = T_R(p; T_A, T_B) \equiv \biggl( \frac{T_A^p + T_B^p}{2} \biggr)^{1/p},
\end{equation}
so named because it takes two thicknesses $T_A$, $T_B$ and ``reduces'' them to a third thickness, similar to a
reduced mass.  The dimensionless parameter $p$ may take any real value in $(-\infty, \infty)$ and is to be
determined by experiment.  This functional form---known as the generalized mean---simplifies to the arithmetic,
geometric, and harmonic mean for certain values of $p$, i.e.
\begin{equation}
  T_R(p; T_A, T_B) =
  \begin{cases}
    \dfrac{T_A + T_B}{2} & p = 1 \text{ (arithmetic)}, \\[2ex]
    \sqrt{T_A T_B} & p = 0 \text{ (geometric)}, \\[2ex]
    \dfrac{2 T_A T_B}{T_A + T_B} & p = -1 \text{ (harmonic)}. \\
  \end{cases}
\end{equation}
More generally, $p$ quantifies the amount of attenuation in asymmetric ($T_A \neq T_B$) regions of the
collision.  By attenuation, we mean \textbf{WHAT?}, as shown in FIG.~\ref{fig:saturation}.  As $p$
\emph{decreases}, the degree of attenuation \emph{increases}.  This significantly impacts the behavior of the
model, for instance $p=1$ is precisely a wounded nucleon model, while $-1 \lesssim p \lesssim 0$ mimics CGC
saturation.

\begin{figure}[t]
  \includegraphics{saturation}
  \caption{
    \label{fig:saturation}
    Attenuation of the reduced thickness.  Thickness $T_B$ is fixed to one (in arbitrary units) while $T_A$ is
    varied.  The corresponding $T_R$ are shown for $p = 1$, 0, $-1$ (blue, green, and red).
  }
\end{figure}

The reduced thickness possesses several other key properties.  It is
\begin{itemize*}
  \item continuous;
  \item monotonically increasing:  \\
    if $T_A \leq T_A'$ and $T_B \leq T_B'$, then $T_R \leq T_R'$;
  \item symmetric in the projectile thicknesses: \\
    $T_R(p; T_A, T_B) = T_R(p; T_B, T_A)$, and hence
  \item independent of $p$ when the thicknesses are equal: \\
    $T_R(p; T, T) = T$.
  \item The reduced thickness vanishes when both $T_A$ and $T_B$ vanish: $T_R(p; 0, 0) = 0$.  In fact, for $p
    \leq 0$, $T_R$ vanishes if \emph{either} $T_A$ or $T_B$ do.  On the other hand, positive values of $p$
    introduce small violations of the eikonal entropy production postulate, similar to a traditional wounded
    nucleon model.
\end{itemize*}

Finally, the reduced thickness provides the basis of the model name:
\trento, for Thickness-Reduced Event-by-event Nuclear Topology.

We now detail the construction of the thickness functions $T_{A,B}(x, y)$, which combined with the definition
of the reduced thickness completes the specification of the model.  The procedure is constructed from the
ground up to handle a variety of collision systems; we begin with the simplest case.

\subsection{Proton-proton collisions}

Consider a collision of two protons $A$, $B$ with impact parameter $b$ along the $x$-direction.
Denote their nuclear densities by
\begin{equation*}
  \rho_{A,B} = \rho_\text{proton}(x \pm b/2, y, z),
\end{equation*}
and let us assume that the integral $\int dz \, \rho_\text{proton}$ either has a closed form or may be
evaluated numerically, so that the proton thickness functions can be calculated.

We now randomly decide whether the protons collide according to the probability \cite{proton-proton}
\begin{equation}
  P_\text{coll} = 1 - \exp\biggl[ -\sigma_{gg} \int dx \, dy \int dz \, \rho_A \int dz \, \rho_B \biggr],
  \label{eq:pcoll}
\end{equation}
where the integral in the exponential is the overlap integral of the proton thickness functions and
$\sigma_{gg}$ is a cross-section which is set so that the total proton-proton cross-section equals the
experimental inelastic nucleon nucleon cross-section $\sigma_{NN}$.

Assuming the protons collide, each is assigned a \emph{fluctuated} thickness
\begin{equation}
  T_{A,B}(x, y) = \gamma_{A,B} \int dz \, \rho_{A,B}(x, y),
\end{equation}
where $\gamma_{A,B}$ are independent random numbers sampled from a gamma distribution with unit mean,
\begin{equation}
  P(\gamma; k) \propto \gamma^{k-1} e^{-\gamma/k},
\end{equation}
with shape parameter $k > 0$ to be fixed by experiment.  The gamma distribution is chosen for its
flexibility---it is exponential for $k = 1$ and becomes Gaussian for large $k$---and because it is the
continuous analog of the negative binomial distribution, which has historically been used to fit proton-proton
multiplicity fluctuations.

The reduced thickness is then calculated from the projectile thickness functions; this furnishes the initial
transverse entropy profile up to an overall normalization factor, $dS/dy \propto T_R(p; T_A, T_B)$.

\subsection{Larger systems}

Composite collision systems such as proton-nucleus and nucleus-nucleus are essentially treated as
superpositions of proton-proton collisions.  A set of nucleon positions is chosen for each
projectile $A$, $B$, then the collision probability \eqref{eq:pcoll} is sampled for each pairwise interaction.
Those nucleons that collide with at least one partner are labeled ``participants'' and the rest are discarded.
The fluctuated thickness function for projectile $A$ is then
\begin{equation}
  T_A = \sum_i \gamma_i \int dz \, \rho_\text{proton}(x_i, y_i, z_i),
\end{equation}
$\gamma_i$ and $(x_i, y_i, z_i)$ are the random fluctuation factor and position, respectively, of participant
$i$ in projectile $A$.  $T_B$ is constructed using the same definition over its participants.

\textbf{SOMETHING HERE?}


\begin{figure*}[t]
  \includegraphics{multdist}
  \caption{
    \label{fig:multdist}
    Minimum bias pp, pPb and PbPb charged particle distributions for power $p=0$, fluctuation parameter
    $k=0.8$ and normalization factor $\kappa'_n$ indicated in the legend.
  }
\end{figure*}


\section{Results}

This section explores the qualitative behavior of the \trento\ model and compares some results to experimental
measurements.  Model calculations depend on several undetermined parameters, namely the reduced thickness
parameter $p$, gamma fluctuation parameter $k$, and nucleon size and shape.  To rigorously constrain these
parameters would require a systematic model-to-data comparison which is beyond the scope of this work;
instead, we invoke some simplifying assumptions.

We restrict our attention to the special cases of the reduced thickness parameter $p = 1$, 0, $-1$ which
correspond respectively to the arithmetic, geometric, and harmonic mean.  The gamma fluctuation parameter $k$
is fixed at 0.8, and we assume the beam-integrated proton density is a Gaussian
\begin{equation}
  \int dz \, \rho_\text{proton} = \frac{1}{2\pi B} \exp\biggr( -\frac{x^2 + y^2}{2B} \biggr)
\end{equation}
with fixed area $B = (0.6\;\text{fm})^2$.  This is at best a rough average profile, so we do not attempt to
explain phenomena such as collective behavior in proton-proton collisions which depend on subnucleonic
structure.

Due to these assumptions, the following results do not necessarily represent the best-fit of the model to data.

\subsection{Multiplicity distributions}

It's been shown that the initial entropy is a close proxy for charged particle multiplicity up
to viscous entropy corrections \needcite
\begin{equation}
  \nch \propto \int dx \, dy \, T_R.
\end{equation}
which allows us to make a direct comparison with experimentally measured charged particle distributions.

\begin{figure}[b]
  \includegraphics{eccentricity}
  \caption{
    \label{fig:eccen}
    Eccentricity harmonics $\varepsilon_2$ (top) and $\varepsilon_3$ (middle) as a function of centrality for
    reduced thickness parameters $p = -1$, 0, 1.  The bottom panel shows the ratio of the rms eccentricities
    $\sqrt{\langle \varepsilon_2^2 \rangle}/\sqrt{\langle \varepsilon_3^2 \rangle}^{\,0.6}$ against the
    experimentally allowed values (grey band) from \cite{constraining-ic}.
  }
\end{figure}

We now use the aforementioned procedure to generate $10^6$ minimum bias events satisfying $N_{part} > 0$ for
proton-proton, proton-lead and lead-lead nuclei using realistic lead nucleon configurations incorporating
short- and long-range correlations \cite{nucleon-correlations}.

In FIG.~\ref{fig:multdist} we plot the predicted charged particle distributions for pp, pPb and PbPb systems
using a geometric mean $p=0$ with fluctuation parameter $k=0.8$ tuned to fit the pp multiplicity
distribution.

The normalization factor $\kappa'_n$ indicated on the figure is tuned to match the average charged particle
multiplicity in each experiment. The observed $\mathcal{O}(20\%)$ discrepancies in $\kappa'_n$ are consistent
with variations in the collision energy and kinematic cuts.

\subsection{Eccentricity harmonics}

In FIG.~\ref{fig:eccen} we plot the PbPb ellipticity and triangularity $\epsilon_2$ and $\epsilon_3$ (top
and middle panels) calculated according to,
\begin{equation}
  \varepsilon_n e^{i n\phi} = -\frac{\int d^2r\, r^n e^{i n \phi} dS/dy(\vec{r})}{\int d^2r\, r^n dS/dy(\vec{r})}.
\end{equation}
along with the eccentricity ratio $\sqrt{\langle \varepsilon_2^2 \rangle}/\sqrt{\langle \varepsilon_3^2
\rangle}^{0.6}$ (bottom panel). The shaded band in the bottom panel indicates the values of this ratio allowed
by LHC flow data and hydrodynamic calculations as determined by the authors in \needcite.

\subsection{Ultracentral uranium-uranium}

It's interesting to compare Eq.~(entropy deposition) against the commonly used wounded nucleon and
binary collision parameterization where the transverse entropy (or energy) density is set proportional to a
linear combination of the local density of participant and pair-wise collisions,
\begin{equation}
  \label{glb ansatz}
  \frac{dS({\bf r}_\perp)}{d^2r_\perp dy} \biggr \rvert_{y=0} \propto
  \frac{1-\alpha}{2}\mbox{WN}({\bf r_\perp}) + \alpha\, \mbox{BC}({\bf r_\perp}).
\end{equation}

In practice, wounded nucleon and binary collision deposition is implemented according to variety of different
prescriptions. We use a two-component model with Gamma fluctuations $\gamma_i$ of the source nucleons given
by,
\begin{equation}
  \label{glb algorithm}
  \frac{dS({\bf r}_\perp)}{d^2r_\perp dy} \biggr \rvert_{y=0} \propto
  \sum\limits_{i=1}^{N_{part}} \gamma_i \left(\frac{1-\alpha}{2} +
  \alpha \frac{N_{coll,i}}{2} \right ) T_p(\vec{r}-\vec{r}_i)
\end{equation}
where $N_{coll,i}$ denotes the number of collisions suffered by the $i$-th participant. The $N_{coll}$ binary
collision term in Eq.~\eqref{glb ansatz} scales with the product of nuclear thickness functions $N_{coll}
\propto T_A T_B$ and predicts a rapid increase in charged particle production in systems for which the density
of nuclear overlap is large.

In contrast with the two-component model, the proposed generalized mean ansatz exhibits no such quadratic
scaling. Specifically, for an idealized system in which the density of nuclear overlap is perfectly symmetric,
i.e.\ $\mathcal{T}_A = \mathcal{T}_B$, the generalized mean and two-component models behave quite differently:
\begin{equation}
  \label{symmetric scaling}
  \frac{dS}{dy} \propto
  \begin{cases}
    \mathcal{T} + \mathcal{C} \mathcal{T}^2 & \text{two-component,} \\
    \mathcal{T} & \text{generalized mean}
  \end{cases}
\end{equation}
where the coefficient $\mathcal{C}$ scales with the binary collision fraction $\alpha$.

An interesting consequence of this non-linearity becomes manifest in central U-U collisions. The Uranium
nucleus is highly deformed and prolonged along a preferred axis. This geometry allows for two distinct
configurations of Uranium collisions in which the density of nuclear matter is fully overlapping - tip-to-tip
collisions where the Uranium symmetry axis is aligned with the beam axis and side-on-side collisions where
the Uranium symmetry axis is perpendicular to the beam axis.

Binary collision scaling predicts that tip-to-tip collisions generate more charged particles than side-on-side
collisions and hence should be found at slightly higher centralities. It was pointed out that this
multiplicity ordering should imprint itself on the flow harmonics of central UU collisions due to the large
ellipticity expected in side-on-side collisions and small ellipticity expected in body-on-body collisions.
This predicted structure was dubbed the ``knee''.

\begin{figure}[t]
  \centering
  \includegraphics{uranium}
  \caption{\label{fig:knee}Scaled TRENTO ($p=0$, $k=0.8$, $B=0.36~\mathrm{fm}^2$) and fluctuated Glauber
  ($\alpha=14$, $k=1.2$, $B=0.36~\mathrm{fm}^2$) ellipticities compared to STAR elliptic flow data.}
\end{figure}

In FIG.~\ref{fig:knee} we compare the predictions of the fluctuated two-component Eq.~\eqref{glb algorithm}
and generalized mean prescriptions Eq.~(entropy deposition) using $10^6$ minimum bias events
in the impact parameter range $0<b<3.5 ~\mathrm{fm}$ with a $15\%$ cut to avoid contamination from events
$b>3.5 ~\mathrm{fm}$.

The fluctuated two component model shows clear evidence of a knee structure in the $0-0.5\%$ centrality bin
which is inconsistent with the STAR data. These findings agree with those previously reported in a
comprehensive flow analysis by A.\ Goldschmidt, Zhi Qiu and U.\ Heinz, in preparation and private communication.
The authors in \needcite\ found that a hot-spot model with sharply peaked entropy density fluctuations described by
concentrated semi-hard cross-section $\sigma \approx 2 ~\mathrm{mb}$ is sufficient to wash out the structure
of the knee in UU collisions, but it remains to be seen if such sharply peaked hot-spots fit into a coherent
description of charged particle production in small systems.

In contrast to the two-component model, the generalized mean correctly reproduces the shape of the
ultra-central elliptic flow measured by STAR. This suggests that non-linearities in the mapping discussed in
Eq.~\eqref{symmetric scaling} are small and it supports attenuation as the mechanism which describes the
centrality dependence of charged particle production in AA collisions.


\section{Summary}

Initial conditions currently comprise the largest source of uncertainty in hydrodynamic simulations of the QGP
fireball [refs]. Ideally, one seeks a first principles calculation of the initial state (i.e.\ see refs), but
these derivations are often irrelevant to the problem of interest making effective models an attractive
alternative.

We propose the family of vector p-norms known as the generalized mean as an effective model for entropy
production in ultra-relativistic nucleus-nucleus collisions and demonstrate the model's ability to
replicate pp, pPb and PbPb multiplicity distributions. In FIG.~\ref{fig:eccen} we show that the predicted
ellipticity of the produced fireball is highly sensitive to the power of the p-norm while the fluctuation
driven triangularity is less sensitive. The optimal ratio of these harmonics which was extracted from flow
data in reference \needcite\ prefers a value of p close to 0 corresponding to the geometric mean. We note that
geometric mean scaling of the initial entropy was also reported by an AdS-CFT calculation in reference
\needcite.

We also compare the generalized mean ansatz against a fluctuated two-component model in Uranium-Uranium
collisions where one expects the eccentricity harmonics to behave quite differently. The generalized mean
correctly predicts the shape of the elliptic flow in ultra-central UU collisions while the two-component model
exhibits a well known knee structure which is inconsistent with the STAR data.


\bibliography{sources}

\end{document}
