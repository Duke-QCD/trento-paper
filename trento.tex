\documentclass[aps,prl,reprint,amsmath,nofootinbib]{revtex4-1}
\usepackage{hyperref}
\usepackage{graphicx}
\graphicspath{{fig/}}


\begin{document}

\title{Entropy production from the generalized mean of nuclear density}

\author{J.\ Scott Moreland}
\author{Jonah E.\ Bernhard}
\author{Steffen A.\ Bass}
\affiliation{Duke University}

\date{\today}


\begin{abstract}
   We investigate entropy deposition at mid-rapidity in pp, pA and AA systems using a generalized mean of projectile and target 
densities $M_p(T_A,T_B)$ which describes a family of vector norms spanning a continuum of projectile/target saturation scenarios. 
Results are presented for three well known subcases of the generalized mean, the arithmetic, geometric and harmonic forms and 
compared against charged particle production in a wide range of collision systems. When combined with partonic density fluctuations of 
the nucleons within each nucleus, the data is well described by a generalized mean with power $p$ close to zero. We encapsulate 
these results in a new model for relativistic hydrodynamic initial conditions.
\end{abstract}

\maketitle

\section{Intoduction}

The discovery of the quark-gluon plasma (QGP) at RHIC initiated a large scale effort to quantify the transport properties of 
hot and dense nuclear matter. The success of ideal hydrodynamics in describing the collectivity observed in ultra-relativistic 
heavy-ion collisions supports the production of a fluid-like QGP with minimal specific shear viscosity close to the proposed 
K.S.S. bound $\eta/s \ge 1/4\pi$ \cite{KSS}. 

Extractions of $\eta/s$ from experiment using viscous hydrodynamics describe the emergence of collective flow from spatial 
deformations of the produced fireball with a conversion efficiency proportional to the system's average specific shear viscosity [?]. 
Consequently, model-to-data predictions of $\eta/s$ are beholden to the veractity of theoretical models used to calculate entropy and 
energy deposition in the initial state.

\section{Motivation}

We consider an ultra-relativistic collision between a projectile nucleus with atomic mass number $A$ and target nucleus with atomic mass number $B$ characterized 
by collision impact parameter ${\bf b}$ in the transverse plane $\vec{r}_\perp = \vec{x} \times \vec{y}$. The density of nuclear matter in the rest frame of each 
nucleus is described by the relevant un-perturbed multi-body nuclear wavefunction $\Psi_N$, where $\Psi_N$ is often characterized by the beam-integrated 
nuclear thickness function,
\begin{equation}
 T(x,y) = \int dz\, \left|\Psi_N(x,y,z) \right|^2. 
\end{equation}

Let $\mathcal{T}_A$ and $\mathcal{T}_B$ denote a sampled transverse density of \emph{participant} nucleons as determined by the collapse of wavefunctions $\Psi_A$ and
$\Psi_B$ in a given nucleus-nucleus collision. We consider the existence of an eikonal mapping $f(\mathcal{T}_A,\mathcal{T}_B)$ that characterizes entropy deposition at mid-rapidity as 
a function of $\mathcal{T}_A$ and $\mathcal{T}_B$,
\begin{equation}
  \label{ansatz}
  \frac{dS({\bf r}_\perp)}{d^2r_\perp dy} \biggr \rvert_{y=0}  = f(\mathcal{T}_A({\bf r}_\perp + {\bf b}/2),\,\mathcal{T}_B({\bf r}_\perp - {\bf b}/2)).
\end{equation}

It's reasonable to insist that $f$ respect several key properties, namely that $f$ is smooth, monatonic and symmetric in the variables $\mathcal{T}_A$ and 
$\mathcal{T}_B$ and that it saturates in the limit $\mathcal{T}_A \ll \mathcal{T}_B$ and $\mathcal{T}_B \ll \mathcal{T}_A$. Motivated by these stipulations, 
we consider the family of vector p-norms known as the generalized mean as a suitable ansatz for the mapping $f$,
\begin{equation}
 \label{generalized mean ansatz}
 f(\mathcal{T}_A, \mathcal{T}_B) \propto M_p(\mathcal{T}_A,\mathcal{T}_B) = \left( \frac{\mathcal{T}_A^p + \mathcal{T}_B^p}{2} \right)^{1/p}.
\end{equation}

For $p = \{-\infty,-1,0,1,\infty\}$, the generalized mean becomes minimum, harmonic, geometric, arithmetic and maximum functions respectively which can be shown
to obey the following ordering inequality, 
\begin{equation}
 M_{-\infty} \le M_{-1} \le M_{0} \le M_{1} \le M_{\infty}.
\end{equation}
In this sense, the continuous parameter $p$ characterizes the dependence of entropy deposition on the projectile-target participant density asymmetry, or equivalently 
the degree of collision saturation. This saturation on $p$ is best visualized by fixing the thickness of the projectile $T_A$ and varying the thickness of the target 
$T_B$ for different values of $p$ as shown in figure \ref{saturation}.
\begin{figure}[b]
 \includegraphics{saturation}
 \caption{\label{saturation}Saturation in the generalized mean $M_p(a, 1)$ for special forms $p=\{-1, 0, 1\}$.}
\end{figure}

\section{The Model}

For simplicity, we refer to the generalized mean of participant projectile and target thicknesses hereafter as the \emph{reduced thickness} of projectile 
nucleus $A$ and target nucleus $B$,
\begin{equation}
 \label{reduced}
 \mathcal{T}_R(p;x,y) \equiv \left( \frac{\mathcal{T}_A({\bf x}+{\bf b}/2,y)^p + \mathcal{T}_B({\bf x}-{\bf b}/2,y)^p}{2} \right)^{1/p}
\end{equation}
noting that the units of $T_R$ are also that of a nuclear thickness function. 

The participant nuclear thickness functions $\mathcal{T}_A$ and $\mathcal{T}_B$ are calculated from an underlying Glauber model of independent nucleon-nucleon 
collisions with additional fluctuations $\gamma_i$ in the transverse density $T_p$ contributed by each participant nucleon,
\begin{equation}
 \label{thickness def}
 \mathcal{T}(x,y) = \sum\limits_{i=1}^{N_{part}} \gamma_i\, T_p(x-x_i,y-y_i).
\end{equation}

The weights $\gamma_i$ are sampled from a two parameter Gamma distribution with mean $\mu=1$ and shape parameter $k$. These weights incorporate event-by-event 
fluctuations in the density of partons sampled in each nucleon-nucleon collision. The choice of a Gamma distribution reflects a long history of fitting proton-proton 
multiplicity distributions using a negative binomial distribution which arises naturally from a Gamma-Poisson mixture.

For the average transverse parton density $T_p(x,y)$ we use a Gaussian,
\begin{equation}
 T_p(x,y) = \frac{1}{\sqrt{2 \pi B}} \exp\left(-\frac{x^2+y^2}{2 B}\right) 
\end{equation}
although we stress that this is a rather crude description of the nucleon and consequently restrict our attention to properties of the collision that are 
insensitive to details of the nucleonic substructure.

Participant nucleons are determined in the projectile and target nuclei using the probabilistic collision criteria described in reference \cite{proton-proton}. In this prescription, the 
probability for a pairwise collision of two nucleons separated by impact parameter $b$ is given by,
\begin{equation}
  \label{coll prob}
  P^{inel}_{NN}(b) = 1 - \exp \left(-\sigma_{gg} N_{gg}^2 T_{pp}(b) \right), \\[1ex]
\end{equation}
where $T_{pp}(b)$ denotes the proton-proton overlap function defined by,
\begin{equation}
 T_{pp}(b) = \int dx~dy \,T_p(x-b/2,y) \,T_p(x+b/2,y).
\end{equation}
The exponential pre-factor $\sigma_{gg} N_{gg}^2$ representing effective glue-glue cross-section and squared gluon count respectively, is tuned to reproduce the total 
inelastic proton-proton cross-section,
\begin{equation}
  \sigma^{inel}_{NN} = \int 2 \pi b \,db \, P_{NN}^{inel}(b).
\end{equation}
The pair-wise collision probability in equation (\ref{coll prob}) is then sampled independently for each pair of projectile and target nucleons to determine the participant nucleons 
which contribute to the summation in equation (\ref{thickness def}).

We finally fix the density of deposited entropy at mid-rapidity according to,
\begin{equation}
 \label{entropy deposition}
 \frac{dS({\bf r}_\perp)}{d^2r_\perp dy} \biggr \rvert_{y=0} = \kappa_s(\sqrt{s})\, \mathcal{T}_R(p;{\bf r}_\perp)
\end{equation}
where $\kappa_s(\sqrt{s})$ is an undetermined energy-dependent normalization factor and $\mathcal{T}_R(x,y)$ is the reduced thickness function.

Equivalently, it's been shown that the initial entropy is a close proxy for charged particle multiplicity up to viscous entropy corrections [?]. We can 
recast equation (\ref{entropy deposition}) in terms of the charged particle production as,
\begin{equation}
 \label{particle production}
 \frac{dN_{ch}({\bf r}_\perp)}{d^2r_\perp dy} \biggr \rvert_{y=0} \approx \kappa_n(\sqrt{s})\, \mathcal{T}_R(p;{\bf r}_\perp)
\end{equation}
which allows us to make a direct comparison with experimentally measured charged particle distributions.

\section{Charged particle distributions}

We show results for proton-proton, proton-lead and lead-lead nuclei using realistic lead nucleon configurations incorporating short- and long-range 
correlations \cite{nucleon-correlations}. For each collision system we generate $10^6$ minimum bias events satisfying $N_{part} > 0$. Charged particle
production at mid-rapidity is calculated from equation (\ref{particle production}) according to, 
\begin{equation}
 N_{ch}(|\eta|<1) = \kappa'_n(\sqrt{s})\, \mathcal{T}_R(p;{\bf r}_\perp)
\end{equation}
where the normalization factor $\kappa'_n$ is scaled to match the average charged particle multiplicity in the pseudo-rapidity interval $|\eta| < 1$.
\begin{figure*}[ht]
    \includegraphics{multdist}
    \caption{\label{fig:multdist}Multiplicity distributions.}
\end{figure*}

In figure \ref{fig:multdist} we illustrate the model performance against the ALICE data using a geometric mean $p=0$ with a fluctuation parameter $k=0.8$ tuned 
to fit the p-p multiplicity distribution and a squared nucleon width fixed at the rather standard value $B=0.36$ $\mbox{fm}^2$. 

The normalization factor $\kappa'_n$ is allowed to vary with each collision system to account for slight differences in the beam energy and kinematic cuts of 
each experiment. The values of the normalization factor and all other model parameters are indicated on the figure.

These parameters do an excellent job capturing the shape of the charged particle distributions with discrepancies in the normalization factor $\kappa'_n$ of
order $\mathcal{O}(20\%)$. It's possible that these discrepancies reflect a sub-optimal choice of model parameters or the absence of subtle physical effects, 
i.e. shadowing of the participant thickness functions $\mathcal{T}_A$ and $\mathcal{T}_B$. We leave these details to a future investigation.


\section{Comparison to a wounded nucleon and binary collision mixture}

It's interesting to compare equation (\ref{entropy deposition}) against the commonly used wounded nucleon and binary collision parameterization where the 
transverse entropy (or energy) density is set proportional to a linear combination of the local density of participant and pair-wise collisions,
\begin{equation}
 \frac{dS({\bf r}_\perp)}{d^2r_\perp dy} \biggr \rvert_{y=0}  \propto \frac{1-\alpha}{2}\mbox{WN}({\bf r_\perp}) + \alpha\, \mbox{BC}({\bf r_\perp}). 
\end{equation}
This two-component ansatz has a variety of incarnations, and we choose the following implementation to make the connection with the generalized mean as meaningful 
as possible,
\begin{equation}
 \label{glb algorithm}
 \frac{dS({\bf r}_\perp)}{d^2r_\perp dy} \biggr \rvert_{y=0}  \propto \sum\limits_{i=1}^{N_{part}} \gamma_i \left(\frac{1-\alpha}{2} + \alpha \frac{N_{coll,i}}{2} \right ) T_p(\vec{r}-\vec{r}_i)
\end{equation}
where $N_{coll,i}$ denotes the number of collisions suffered by the $i$-th participant and $\gamma_i$ is sampled from a Gamma distribution as before.

One can easily check from equations (\ref{entropy deposition}) and (\ref{glb algorithm}) that both the reduced thickness function the two-component model reduce to a 
fluctuated wounded nucleon model for $p=1$ and $\alpha=0$ respectively. This is however, where the simularities between the two models end.

For example along the symmetry axis $T_A=T_B$ the mappings behave quite differently. The series expansion of the reduced thickness function is linear in $T$,
\begin{equation}
 \mathcal{T}_R(T,T) \approx C_1 T
\end{equation}
while the series expansion of the two-component model denoted by the mapping $\mathcal{G}(T_A,T_B)$ includes terms that are quadratic in $T$,
\begin{equation}
 \mathcal{G}(T,T) \approx C_1 T + C_2 T^2.
\end{equation}

An interesting consequence of this non-linearity becomes manifest in central U-U collisions. The Uranium nucleus is highly deformed and prolonged along a preferred 
axis. This geometry allows for two distinct configurations of Uranium collisions in which the density of nuclear matter is fully overlapping - tip-to-tip collisions 
where the Uranium symmetry axis is alligned with the beam axis and body-on-body collisions where the Uranium symmetry axis is perpendicular to the beam axis. 

Tip-to-tip configurations generate more binary collisions than body-on-body configurations and consequently, binary collision scaling would predict that these 
collisions produce more charged particles. This multiplicity ordering of tip-to-tip and body-on-body configurations creates a surplus of highly elliptic 
body-on-body configurations just below the most central event class which is predominantly tip-to-tip. It was shown by the authors in [?] that this sorting predicts 
a curious knee shaped structure in the centrality dependent elliptic flow data. 

Recent $v_2\{2\}$ and $v_2\{4\}$ flow results from STAR found no such evidence of a knee-shaped structure in ultra-central U-U collisions \cite{UU-STAR}. It has been
suggested by the authors in [?] suggested that the existence of hot spots in the produced fireball may wash out the structure of the knee, the analysis model with a higher degree of fluctuations,

This suggests that non-linearities in the symmetric component of equation (\ref{ansatz}) are small and thus the symmetric scaling $M_p(T,T) \propto T$ employed by the generalized mean
is a reasonable approximation.


\section{Summary}

\bibliography{sources}

\end{document}
