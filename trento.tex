\documentclass[aps,prl,reprint,amsmath,nofootinbib]{revtex4-1}

\usepackage{hyperref}
\usepackage{graphicx}
\usepackage{afterpage}
\graphicspath{{fig/}}

\usepackage{mdwlist}
\renewcommand\labelitemi{\raisebox{.3ex}{\tiny$\bullet$}}

\newcommand{\trento}{T\raisebox{-.5ex}{R}ENTo}
\newcommand{\nch}{N_\text{ch}}
\newcommand{\eccratio}{\sqrt{\langle \varepsilon_2^2 \rangle}/\sqrt{\langle \varepsilon_3^2 \rangle}^{\,0.6}}
\newcommand{\needcite}{\textbf{[???]}}

\begin{document}

\title{A unified, effective model for particle production in high-energy nuclear collisions}

\author{J.\ Scott Moreland}
\author{Jonah E.\ Bernhard}
\author{Steffen A.\ Bass}
\affiliation{Department of Physics, Duke University, Durham, NC 27708-0305}

\date{\today}


\begin{abstract}
  The new \trento\ initial condition model is assembled using an effective description of entropy production
  in high-energy nuclear collisions.  Self-consistent pp, pPb, and PbPb multiplicity distributions and PbPb
  eccentricity harmonics are calculated and compare favorably to experiment.  A possible explanation of recent
  uranium-uranium results is proposed.
\end{abstract}


\maketitle

\section{Introduction}

Over the last decade, the ultra-relativistic heavy-ion collision programs at the Relativistic Heavy Ion Collider (RHIC) and the Large Hadron Collider 
(LHC) have succeeded in producing and exploring a novel, highly excited phase of QCD matter dubbed the strongly interacting Quark-Gluon Plasma (sQGP) 
\cite{Arsene:2004fa,Adcox:2004mh,Back:2004je,Adams:2005dq,Gyulassy:2004zy,Muller:2006ee,Muller:2012zq}. Among the major goals of current research into 
the sQGP are the quantification of its properties, such as transport coefficients and equation of state, and exploration of the non-equilibrium dynamics 
that lead to its formation in the first fm/c of a nuclear collision. 

Viscous relativistic fluid dynamics provides 
a stable, well-tested description of this time-evolution and the collectivity observed in high-energy nuclear collisions \cite{}.  However, a 
realistic Monte Carlo (MC) model for the initial state of the medium is essential for describing many event-by-event observables.  These initial 
conditions remain one of the most poorly-constrained components of modern computational models \cite{}.

Initial condition models create profiles of energy or entropy at the time of sQGP formation which are then evolved by
fluid dynamics.  Perhaps the simplest prescription is the so-called wounded nucleon model, which determines
participating nucleons via optical overlap and deposits a blob of energy or entropy for each participant.  The
MC-Glauber model generalizes wounded nucleons by additionally depositing entropy proportional to the number of
binary nucleon-nucleon collisions.  Despite its simplicity, the Glauber model has qualitatively fit a large number 
of experimental measurements \cite{}.

The current state-of-the-art for initial condition models is the IP-Glasma model \cite{}, which is based on the color-glass condensate (CGC) effective 
field theory \cite{} to create a dynamical description of the pre-fluid stage of the collision.  IP-Glasma has been able to quantitatively describe the 
latest event-by-event data on higher order flow harmonics as well as many other observables \cite{Schenke:2014zha}. It's main draw back is it's computational 
complexity and that it's applicability to smaller collision systems (e.g.~proton-proton) remains uncertain.

Most recently, members of the PHENIX collaboration demonstrated that a wounded quark model can be used to describe the transverse energy distributions of 
proton-proton, deuteron-gold and gold-gold collisions at RHIC without the use of a binary collision term. It is however, not yet clear if these scaling laws are
valid at the LHC where gluon occupation is expected to dominate.

In this letter, we present a new initial condition model, \trento, for high-energy nuclear collisions spanning the full range of projectile and target combinations, 
from proton-proton to proton-lead, lead-lead and uranium-uranium collisions. In concert with transport calculations for the time-evolution of ultra-relativistic 
nuclear collisions, such an approach provides {\bf PUT SOMETHING HERE} %a unified description of high energy nucleus-nucleus collisions.

\section{The Model}

\trento\ is constructed in a similar fashion as the two-component Glauber model, but uses a different ansatz
for entropy production. 

Suppose a pair of projectiles labeled $A$, $B$ collide along beam axis $z$.  Each projectile is represented by
the beam-integrated density of its nuclear matter, typically called thickness:
\begin{equation}
  T_{A,B}(x, y) = \int dz \, \rho_{A,B}(x, y, z).
\end{equation}
The construction of the thickness functions will be addressed in the following subsections; first, we
postulate the following:
\begin{enumerate}
  \item The eikonal approximation is valid:  entropy is produced if $T_A$ and $T_B$ eikonally overlap.
  \item There exists a scalar field $f(T_A, T_B)$ which converts projectile thicknesses into entropy
    deposition.
\end{enumerate}
The function $f$ is proportional to the entropy created at mid-rapidity and at the hydrodynamic thermalization
time:
\begin{equation}
  f \propto dS/dy \, |_{\tau = \tau_0}.
\end{equation}
It should provide an \emph{effective} description of early collision dynamics:  it need not arise from a
first-principles calculation, but it must obey basic physical constraints.

With this in mind, we introduce for $f$ the \emph{reduced thickness}
\begin{equation}
  f = T_R(p; T_A, T_B) \equiv \biggl( \frac{T_A^p + T_B^p}{2} \biggr)^{1/p},
  \label{eq:tr}
\end{equation}
so named because it takes two thicknesses $T_A$, $T_B$ and ``reduces'' them to a third thickness, similar to a
reduced mass.  The dimensionless parameter $p$ may take any real value in $(-\infty, \infty)$ and is to be
determined by experiment.  This functional form --- known as the generalized mean or vector p-norm --- simplifies to arithmetic,
geometric, and harmonic means for certain values of $p$, i.e.
\begin{equation}
  T_R(p; T_A, T_B) =
  \begin{cases}
    \dfrac{T_A + T_B}{2} & p = 1 \text{ (arithmetic)}, \\[2ex]
    \sqrt{T_A T_B} & p = 0 \text{ (geometric)}, \\[2ex]
    \dfrac{2\, T_A T_B}{T_A + T_B} & p = -1 \text{ (harmonic)}. \\
  \end{cases}
\end{equation}
More generally, $p$ quantifies the attenuation of entropy production in asymmetric ($T_A \neq T_B$) regions of
the collision, as demonstrated in FIG.~\ref{fig:saturation}.  As $p$ \emph{decreases}, the degree of
attenuation \emph{increases}.  This significantly impacts the behavior of the model, for instance $p=1$ is
precisely a wounded nucleon model, while $-1 \lesssim p \lesssim 0$ mimics features of CGC saturation based models.

The reduced thickness possesses several other key properties.  It is
\renewcommand{\labelenumi}{(\alph{enumi})}
\begin{enumerate*}
  \item continuous and monotonically increasing, \\ if $T_A \leq T_A'$ and $T_B \leq T_B'$, then $T_R \leq T_R'$;
  \item symmetric in the projectile thicknesses, \\ $T_R(p; T_A, T_B) = T_R(p; T_B, T_A)$;
  \item independent of $p$ when the thicknesses are equal, \\ $T_R(p; T, T) = T$ and hence
  \item vanishes when $T_A$ and $T_B$ vanish $T_R(p; 0, 0) = 0$. \\ In fact, for $p \leq 0$, $T_R$ vanishes 
     if \emph{either} $T_A$ or $T_B$ do.  On the other hand, positive values of $p$ introduce small violations 
    of the eikonal entropy production postulate, similar to a traditional wounded nucleon model.
\end{enumerate*}

Finally, the reduced thickness provides the basis of the model name:
\trento, for Thickness-Reduced Event-by-event Nuclear Topology.

We now detail the construction of the thickness functions $T_{A,B}(x, y)$, which combined with the definition
of the reduced thickness completes the specification of the model.  The procedure is constructed from the
ground up to handle a variety of collision systems; we begin with the simplest case.

\begin{figure}[t]
  \includegraphics{saturation}
  \caption{
    \label{fig:saturation}
    Attenuation of the reduced thickness.  Thickness $T_B$ is fixed to one (in arbitrary units) while $T_A$ is
    varied.  The corresponding $T_R$ are shown for $p = 1$, 0, $-1$ (green, blue, and red).
  }
\end{figure}

\begin{figure*}[t]
  \includegraphics{multdist}
  \caption{
    \label{fig:multdist}
    Multiplicity distributions for proton-proton, proton-lead, and lead-lead collisions.  The blue histograms
    are \protect\trento\ results from $10^6$ minimum-bias events for each collision system, all with reduced
    thickness parameter $p = 0$ (geometric mean) and gamma fluctuation parameter $k = 0.8$.  The normalization
    constants indicated in the legends are set so that the model and experimental distributions (points with
    error bars) have the same mean.  Experimental data are from ALICE \needcite.
  }
\end{figure*}

\subsection{Proton-proton collisions}

Let us consider a collision of two protons $A$, $B$ with impact parameter $b$ along the $x$-direction.
We denote their nuclear densities by
\begin{equation}
  \rho_{A,B} = \rho_\text{proton}(x \pm b/2, y, z),
\end{equation}
and assume that the integral $\int dz \, \rho_\text{proton}$ either has a closed form or may be
evaluated numerically, so that the proton thickness functions can be calculated.

We now randomly sample the collision probability of the protons according to \cite{dEnterria:2010hd},
\begin{equation}
  P_\text{coll} = 1 - \exp\biggl[ -\sigma_{\mathrm{partonic}} \int dx \, dy \int dz \, \rho_A \int dz \, \rho_B \biggr],
  \label{eq:pcoll}
\end{equation}
where the integral in the exponential is the overlap integral of the proton thickness functions and
$\sigma_{\mathrm{partonic}}$ is a parton-parton cross-section which is tuned such that the total proton-proton 
cross-section in our model equals the experimentally measured inelastic nucleon nucleon cross-section $\sigma_{NN}$.

Assuming the protons collide, each is assigned a \emph{fluctuated} thickness
\begin{equation}
  T_{A,B}(x, y) = \gamma_{A,B} \int dz \, \rho_{A,B}(x, y),
\end{equation}
where $\gamma_{A,B}$ are independent random numbers sampled from a gamma distribution with unit mean,
\begin{equation}
  P(\gamma; k) \propto \gamma^{k-1} e^{-\gamma/k},
\end{equation}
with the shape parameter $k > 0$ to be fixed by experiment.  The gamma distribution is chosen for its
flexibility -- it is exponential for $k = 1$ and becomes Gaussian for large $k$ -- and because it is the
continuous analog of the negative binomial distribution, which has historically been used to fit proton-proton
multiplicity fluctuations.

The reduced thickness is then calculated from the projectile thickness functions; this furnishes the initial
transverse entropy profile up to an overall normalization factor, 
\begin{equation}
dS/dy \, |_{\tau = \tau_0} \propto T_R(p; T_A, T_B).
\end{equation}

\subsection{Larger systems p+A and A+A}

Composite collision systems such as proton-nucleus and nucleus-nucleus are essentially treated as
superpositions of proton-proton collisions.  A set of nucleon positions is sampled for each
projectile $A$, $B$, utilizing either an uncorrelated Woods-Saxon distribution or correlated nuclear configurations 
when available \cite{Alvioli:2009ab}. Subsequently the collision probability \eqref{eq:pcoll} is sampled for each 
pairwise interaction. Those nucleons that collide with at least one partner are labeled ``participants'' and the 
rest are discarded. The fluctuated thickness function of nucleus $A$ then reads
\begin{equation}
  T_A = \sum_i \gamma_i \int dz \, \rho_\text{proton}(x - x_i, y - y_i, z - z_i),
  \label{nuclear thickness}
\end{equation}
where $\gamma_i$ and $(x_i, y_i, z_i)$ are the random fluctuation factor and position, respectively, of
participant $i$ in nucleus $A$. Fluctuated thickness function $T_B$ follows analygously. 

\begin{figure*}[t]
  \includegraphics{eccentricity}
  \caption{
    \label{fig:eccen}
    Left and middle plots:  Eccentricity harmonics $\varepsilon_2$ and $\varepsilon_3$ as a function of centrality 
    for reduced thickness parameters $p = 1$, 0, $-1$ (green, blue, and red).  Right plot:  Ratio of the rms eccentricities 
    $\eccratio$ against the experimentally allowed region (grey band) from \cite{Retinskaya:2013gca}.  Note that the centrality 
    axis has a different range in the ratio plot.
  }
\end{figure*}

At this point, it's useful to enumerate the parameters of the model and summarize its construction. 
The inputs to the model are the Monte-Carlo nucleon coordinates $\{\vec{x}_1,\vec{x}_2,...,\vec{x}_{A,B}\}$, a randomly sampled 
nucleus-nucleus impact parameter $b$, the nucleon-nucleon cross section $\sigma_{NN}$, the nucleon density profile 
$\rho_{\mathrm{proton}}$, the Gamma shape parameter $k$ in the fluctuated nuclear thickness function, the generalized 
mean exponent $p$ in the definition of the \emph{reduced} thickness function, and an overall normalization factor denoted simply as ``norm''.

The output of the model is the transverse entropy density in an interval about mid-rapidity constructed from the reduced thickness function 
$dS/dy \propto T_R(p;T_A,T_B)$ using expression \eqref{eq:tr}. We now proceed to demonstrate the effectiveness of the model in 
describing a wide range of collision systems.

\section{Model Applications}

\trento\ simulations are designed with an inherrent freedom in the choice of several parameters such as the nucleon density profile $\rho_\mathrm{proton}$, Gamma fluctuation parameter $k$,
and generalized mean exponent $p$. Rigorously constraining these parameters would require a systematic model-to-data comparison which is beyond the scope of this work.  
Therefore, the following results do not necessarily represent the best-fit of the model to data.

\subsection{Multiplicity distributions}

The experimentally observed charged-particle multiplicity $\nch$ is to a very good approximation proportional to the total initial
entropy \needcite, and hence proportional to the integrated reduced thickness in our model:
\begin{equation}
  \nch \propto \int dx \, dy \, T_R.
\end{equation}
To compare with experiment, we generate a large ensemble of minimum-bias events, integrate their $T_R$
profiles, and determine a multiplicative normalization constant in order to rescale the distribution to the experimental mean.

The left panel of FIG.~\ref{fig:multdist} shows the $\nch$ distribution from $10^6$ proton-proton simulations
using reduced thickness parameter $p = 0$ (geometric mean), gamma fluctuation parameter $k = 0.8$, and
Gaussian beam-integrated proton density
\begin{equation}
  \int dz \, \rho_\text{proton} = \frac{1}{2\pi B} \exp\biggr( -\frac{x^2 + y^2}{2B} \biggr)
\end{equation}
with an effective area of $B = (0.6\;\text{fm})^2$.  The results compare favorably with the experimental distribution
measured by ALICE \needcite.

Proton-lead and lead-lead distributions are presented in the middle and right panels of
FIG.~\ref{fig:multdist}. The Model parameters are  identical to proton-proton, except for the overall normalization factor, which depends on the beam energy and kinematic cuts, both of which vary across the data analyzed in this figure.
Overall, the model is able to reproduce the shapes of all three distributions using a single set of parameters, and only allowing for a change in the normalization, whose $\sim$20\% variation across collision systems is consistent with differences in experimental beam energy and kinematic cuts.

\subsection{Eccentricity harmonics}

Eccentricity harmonics $\varepsilon_n$ can be calculated using the definition
\begin{equation}
  \varepsilon_n e^{i n\phi} = -\frac{\int dx \, dy\, r^n e^{i n \phi} \, T_R}{\int dx \, dy \, r^n \, T_R}.
\end{equation}
Figure~\ref{fig:eccen} shows ellipticity $\varepsilon_2$ and triangularity $\varepsilon_3$ for Pb+Pb collisions as a function of
centrality for reduced thickness $p = 1$,~0,~$-1$.  There is a clear trend of increasing eccentricity
(particularly $\varepsilon_2$) with decreasing $p$.  This may be understood by the attenuation property of the
reduced thickness:  as $p$ decreases, asymmetric regions of the collision produce less entropy, which
accentuates the elliptical overlap shape in non-central collisions and enhances their eccentricity.

In addition, we perform the test described in \cite{Retinskaya:2013gca} and calculate the ratio of root-mean-square eccentricities
$\eccratio$.  In \cite{Retinskaya:2013gca}, the authors have used experimental flow data
and hydrodynamics to construct an experimentally allowed band of the ratio of root-mean-square eccentricities
as a function of centrality.  They compare a number of common initial condition models to that band and find that
currently only the IP-Glasma model consistently falls within the allowed region.  As can be seen in the bottom panel of FIG.~\ref{fig:eccen} our model yields excellent agreement with the allowed band for
the geometric mean ($p = 0$).

\subsection{Ultracentral U+U collisions}

\begin{figure}[b]
  \centering
  \includegraphics{uranium}
  \caption{
    \label{fig:uranium}
    Ellipticity $\varepsilon_2$ as a function of normalized charged-particle multiplicity
    $\nch/\langle\nch\rangle$ in ultracentral uranium-uranium and gold-gold collisions at RHIC.  The top and
    bottom plots show the top 0.1\% and 1\% (respectively) of collisions selected by number of spectators to
    mimic STAR's experimental ZDC selection \cite{FortheSTAR:2013bza}.  Blue points with error bars are binned
    \protect\trento\ results from $10^6$ events with reduced thickness parameter $p = 0$ (geometric mean) and
    gamma fluctuation parameter $k = 0.8$.  Blue lines are linear fits within
    $0.9~<~\nch/\langle\nch\rangle~<~1.1$.  Grey lines represent the analogous Glauber+NBD slopes calculated
    in \cite{FortheSTAR:2013bza}.
  }
\end{figure}

Uranium nuclei have a prolate, spheroidal shape which can be exploited to achieve a larger variance in the anisotropy of thermalized QGP matter. 
These deformed nuclear collisions achieve maximal overlap via two distinct orientations: ``tip-tip'' collisions, in which the long axes of the spheroids are 
aligned with the beam axis and the overlap area is circular; or ``side-side'' collisions, where the long axes are perpendicular to the beam axis and the overlap area is elliptical. 

In the two-component Glauber model, tip-tip collisions have the same number of wounded nucleons as side-side collisions while generating more binary collisions and hence more charged particles.
Consequently, Glauber models predict an excess of tip-tip collisions in ultra-central event classes where binary collision scaling dominates. These tip-tip collisions 
are characterized by highly circular overlap regions and thus are expected to produce very little elliptic flow. This predicted drop in the uranium-uranium elliptic flow harmonic is known as the ``knee'' \needcite.

The STAR experiment at RHIC recently performed uranium-uranium collisions and found no evidence that tip-tip collisions produce more charged particle than side-side collisions and no evidence of the 
a knee-like structure in the measured flow harmonics \cite{FortheSTAR:2013bza}.  It has been proposed that fluctuations could wash out the knee \needcite, but a recent
comprehensive flow analysis showed that the knee would still be visible \cite{osu}.

The reduced thickness ansatz in our model \eqref{eq:tr}, however, does not exhibit binary collision scaling and hence predicts similar slopes for ultra-central 
gold-gold and uranium-uranium collisions as shown in FIG.~\ref{fig:uranium}. Short of conducting a full hydrodynamic analysis, our model results therefore appear to be more consistent 
with STAR data than Glauber, and are similar to calculations by the IP-Glasma model \needcite.


\section{Conclusion}

We introduce \trento, a new unified initial condition model for high-energy nuclear collisions based on
eikonal entropy creation via the ``reduced thickness''.  The model is able to simultaneously fit
proton-proton, proton-lead, and lead-lead multiplicity distributions along with lead-lead eccentricity
harmonics, and provides a possible resolution to the ``knee'' problem in ultracentral uranium-uranium
collisions.  It is flexible by design and able to mimic the behavior of other initial condition models.

This is an intentionally \emph{effective} model.  A first-principles derivation would be ideal, but such
models are usually complex and require significant computation time.  So in practical terms, effective models
are often preferable for empirical modeling.

Future work will couple the initial condition generator to a full hydrodynamical simulation and constrain the
various model parameters.  We plan to publicly release \trento\ in the coming weeks, and the community is
encouraged to use, test, and contribute to the source code.

\section*{Acknowledgements}

We would like to thank Ulrich Heinz and Berndt M\"uller for many helpful discussions and valuable feedback. 
SAB is being supported by the U.S. Department of Energy Grant no. DE-FG02-05ER41367, JB is supported through the NSF Office of Cyberinfrastructure by grant no. PHY-0941373 and JSM acknowledges support by the DOE/NNSA Stockpile Stewardship Graduate Fellowship under 
grant no. DE-FC52-08NA28752.

\bibliography{trento,duke-qcd-refs/Duke_QCD_refs}


\end{document}
