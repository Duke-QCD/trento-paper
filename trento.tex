\documentclass[aps,prl,reprint,amsmath,nofootinbib]{revtex4-1}
\usepackage{hyperref}
\usepackage{graphicx}
\usepackage{subfig}
\graphicspath{{fig/}}


\begin{document}

\title{Entropy production from the generalized mean of nuclear density}

\author{J.\ Scott Moreland}
\author{Jonah E.\ Bernhard}
\author{Steffen A.\ Bass}
\affiliation{Duke University}

\date{\today}


\begin{abstract}
   We investigate entropy deposition at mid-rapidity in pp, pA and AA systems using a generalized mean of projectile and target 
densities $M_p(T_A,T_B)$ which describes a family of vector norms spanning a continuum of projectile/target saturation scenarios. 
Results are presented for three well known subcases of the generalized mean, the arithmetic, geometric and harmonic forms and 
compared against charged particle production in a wide range of collision systems. When combined with partonic density fluctuations of 
the nucleons within each nucleus, the data is well described by a generalized mean with power $p$ close to zero. We encapsulate 
these results in a new model for relativistic hydrodynamic initial conditions.
\end{abstract}

\maketitle

\section{Intoduction}

The discovery of the quark-gluon plasma (QGP) at RHIC initiated a large scale effort to quantify the transport properties of 
hot and dense nuclear matter. The success of ideal hydrodynamics in describing the collectivity observed in ultra-relativistic 
heavy-ion collisions supports the production of a fluid-like QGP with minimal specific shear viscosity close to the proposed 
K.S.S. bound $\eta/s \ge 1/4\pi$ \cite{constraining-ic}. 

Extractions of $\eta/s$ from experiment using viscous hydrodynamics describe the emergence of collective flow from spatial 
deformations of the produced fireball with a conversion efficiency proportional to the system's average specific shear viscosity [?]. 
Consequently, model-to-data predictions of $\eta/s$ are beholden to the veractity of theoretical models used to calculate entropy and 
energy deposition in the initial state.

\section{Motivation}

We characterize entropy deposition in nucleus-nucleus collisions in terms of an eikonal mapping defined over the transverse density 
of projectile and target nucleons $T_A$ and $T_B$ separated by impact parameter ${\bf b}$,
\footnote{We choose entropy density as a convenience to make direct comparisons with charged particle distributions}
\begin{equation}
  \frac{dS({\bf r}_\perp)}{d^2r_\perp dy} \biggr \rvert_{y=0}  = f(T_A({\bf r}_\perp + {\bf b}/2),\,T_B({\bf r}_\perp - {\bf b}/2)).
\end{equation}

It's reasonable to insist that $f$ respect several key properties, namely that $f$ is smooth, monatonic and symmetric in 
the variables $T_A$ and $T_B$ and that it saturates in the limit $T_A \ll T_B$ and $T_B \ll T_A$.

Motivated by these stipulations, we consider the family of vector p-norms known as the generalized mean as a suitable ansatz for the mapping $f$,
\begin{equation}
 f(T_A, T_B) \propto M_p(T_A,T_B) 
\end{equation}
where the generalized mean is defined as,
\begin{equation}
 \label{generalized mean}
 M_p(T_A,T_B) = \left( \frac{T_A^p + T_B^p}{2} \right)^{1/p}.
\end{equation}

For $p = \{-\infty,-1,0,1,\infty\}$, the generalized mean becomes minimum, harmonic, geometric, arithmetic and maximum functions respectively. Increasing 
values of $p$ result in decreasing saturation so that these subcases of the generalized mean obey the following ordering inequality, 
\begin{equation}
 M_{-\infty} \le M_{-1} \le M_{0} \le M_{1} \le M_{\infty}.
\end{equation}
The dependence of the saturation on $p$ is best visualized by fixing the thickness of the projectile $T_A$ and varying the thickness of the target $T_B$ as
shown in figure [1].
\begin{figure}[b]
 \includegraphics[width=\linewidth]{saturation.pdf}
 \caption{Saturation in the generalized mean $M_p(1,T_B)$ for special forms $p=\{-\infty, -1, 0, 1, \infty\}$.}
\end{figure}

It's worth mentioning that the generalized mean is equivalent to a wounded nucleon model for $p=1$ such that $M_1(T_A,T_B) \propto T_A + T_B$. However, the
generalized mean is a homogenous function of $T_A$ and $T_B$ and there are no terms equivalent to the density of binary collisions which scale proportional
to the product $T_A T_B$.


\section{The Model}
We sample nucleon coordinates in the projectile and target nuclei using the realistic nucleon configurations calculated by Alvioli 
and Strickland [?].

We follow the precription of [ref] and calculate pairwise collisions of nucleons in pp, pA and AA collisions from the proton-proton 
overlap function,
\begin{align}
 T_{pp}(b) &= \int dx~dy \,T_p(x-b/2,y) \,T_p(x+b/2,y),
\end{align}
where $b$ denotes the proton-proton impact parameter and $T_p(x,y)$ the proton's z-integrated parton density.

The probability for a pairwise collision is given by,
\begin{equation}
  P^{inel}_{NN}(b) = 1 - \exp \left(-\sigma_{gg} N_{gg}^2 T_{pp}(b) \right), \\[1ex]
\end{equation}
where the exponential factor $\sigma_{gg} N_{gg}^2$ is fixed to reproduce the inelastic proton-proton cross section,
\begin{equation}
  \sigma^{inel}_{NN} = \int 2 \pi b \,db \, P_{NN}^{inel}(b).
\end{equation}

The pair-wise collision probability is sampled independently for each pair of nucleons in the collision and target

\section{Charged particle distributions}

\begin{figure*}
    \centering
    \subfloat[label 1]{\includegraphics[width=0.333\textwidth]{pp.pdf}}
    \subfloat[label 2]{\includegraphics[width=0.333\textwidth]{pPb.pdf}}
    \subfloat[label 3]{\includegraphics[width=0.333\textwidth]{PbPb.pdf}}
    \caption{2 Figures side by side}
    \label{fig:example}
\end{figure*}

\bibliography{sources}


\end{document}
