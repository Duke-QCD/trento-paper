\documentclass[aps,prl,reprint,amsmath,nofootinbib]{revtex4-1}
\usepackage{hyperref}
\usepackage{graphicx}
\usepackage{subfig}
\graphicspath{{fig/}}


\begin{document}

\title{Entropy production from the generalized mean of nuclear density}

\author{J.\ Scott Moreland}
\author{Jonah E.\ Bernhard}
\author{Steffen A.\ Bass}
\affiliation{Duke University}

\date{\today}


\begin{abstract}
   We investigate entropy deposition at mid-rapidity in pp, pA and AA systems using a generalized mean of projectile and target 
densities $M_p(T_A,T_B)$ which describes a family of vector norms spanning a continuum of projectile/target saturation scenarios. 
Results are presented for three well known subcases of the generalized mean, the arithmetic, geometric and harmonic forms and 
compared against charged particle production in a wide range of collision systems. When combined with partonic density fluctuations of 
the nucleons within each nucleus, the data is well described by a generalized mean with power $p$ close to zero. We encapsulate 
these results in a new model for relativistic hydrodynamic initial conditions.
\end{abstract}

\maketitle

\section{Intoduction}

The discovery of the quark-gluon plasma (QGP) at RHIC initiated a large scale effort to quantify the transport properties of 
hot and dense nuclear matter. The success of ideal hydrodynamics in describing the collectivity observed in ultra-relativistic 
heavy-ion collisions supports the production of a fluid-like QGP with minimal specific shear viscosity close to the proposed 
K.S.S. bound $\eta/s \ge 1/4\pi$ \cite{constraining-ic}. 

Extractions of $\eta/s$ from experiment using viscous hydrodynamics describe the emergence of collective flow from spatial 
deformations of the produced fireball with a conversion efficiency proportional to the system's average specific shear viscosity [?]. 
Consequently, model-to-data predictions of $\eta/s$ are beholden to the veractity of theoretical models used to calculate entropy 
and energy deposition in the initial state.

\section{Motivation}

We characterize the hydrodynamic initial conditions in terms of an eikonal mapping $f(T_A,T_B)$, that converts a local transverse 
density of partons in the projectile and target $T_A(x,y)$ and $T_B(x,y)$ into a deposition of entropy at mid-rapidity 
\footnote{We choose entropy density as a convenience to make direct comparisons with charged particle distributions}
\begin{equation}
  \frac{dS({\bf r}_\perp)}{d^2r_\perp dy} \biggr \rvert_{y=0}  = f(T_A({\bf r}_\perp),T_B({\bf r}_\perp)).
\end{equation}
It's reasonable for this mapping to respect several key properties, namely that $f(T_A,T_B)$ is smooth, monatonic and symmetric in 
the variables $T_A$ and $T_B$. It's also reasonable to expect that it saturates in the limit $\min(T_A,T_B) \ll \max(T_A,T_B)$.

Motivated by these stipulations, we consider the family of vector p-norms on the two-component thickness vector $\vec{T} = (T_A,T_B)$ 
as a suitable ansatz for $f(T_A,T_B)$,
\begin{equation}
 \label{generalized mean}
 f(T_A,T_B) \propto M_p(T_A,T_B) = \left( \frac{T_A^p + T_B^p}{2} \right)^{1/p}.
\end{equation}

The continuous parameter $p$ in equation (\ref{generalized mean}) interpolates between the well known vector norms:
\begin{flalign*} 
 M_{-\infty}(T_A,T_B) &= \min(T_A,T_B) &\text{(minimum)} \\
 M_{-1}(T_A,T_B) &= 2\,T_A T_B / (T_A + T_B) &\text{(harmonic)} \\
 M_{0}(T_A,T_B) &= \sqrt{T_A T_B} &\text{(geometric)} \\
 M_{1}(T_A,T_B) &= (T_A+T_B)/2 &\text{(arithmetic)} \\
 M_{\infty}(T_A,T_B) &= \max(T_A,T_B). &\text{(maximum)}
\end{flalign*}

Increasing values of $p$ result in decreasing saturation so that the set of p-norms obey the ordering inequality, 
\begin{equation}
 M_{-\infty}(\vec{T}) \le M_{-1}(\vec{T}) \le M_{0}(\vec{T}) \le M_{1}(\vec{T}) \le M_{\infty}(\vec{T}).
\end{equation}

\begin{figure}[b]
 \includegraphics[width=\linewidth]{saturation.pdf}
 \caption{Saturation in the generalized mean $M_p(1,T_B)$ for special forms $p=\{-\infty, -1, 0, 1, \infty\}$.}
\end{figure}


\section{The Model}
We sample nucleon coordinates in the projectile and target nuclei using the realistic nucleon configurations calculated by Alvioli 
and Strickland [?].

We follow the precription of [ref] and calculate pairwise collisions of nucleons in pp, pA and AA collisions from the proton-proton 
overlap function,
\begin{align}
 T_{pp}(b) &= \int dx~dy \,T_p(x-b/2,y) \,T_p(x+b/2,y),
\end{align}
where $b$ denotes the proton-proton impact parameter and $T_p(x,y)$ the proton's z-integrated parton density.

The probability for a pairwise collision is given by,
\begin{equation}
  P^{inel}_{NN}(b) = 1 - \exp \left(-\sigma_{gg} N_{gg}^2 T_{pp}(b) \right), \\[1ex]
\end{equation}
where the exponential factor $\sigma_{gg} N_{gg}^2$ is fixed to reproduce the inelastic proton-proton cross section,
\begin{equation}
  \sigma^{inel}_{NN} = \int 2 \pi b \,db \, P_{NN}^{inel}(b).
\end{equation}

The pair-wise collision probability is sampled independently for each pair of nucleons in the collision and target

\section{Charged particle distributions}

\begin{figure*}
    \centering
    \subfloat[label 1]{\includegraphics[width=0.333\textwidth]{pp.pdf}}
    \subfloat[label 2]{\includegraphics[width=0.333\textwidth]{pPb.pdf}}
    \subfloat[label 3]{\includegraphics[width=0.333\textwidth]{PbPb.pdf}}
    \caption{2 Figures side by side}
    \label{fig:example}
\end{figure*}

\bibliography{sources}


\end{document}
