\documentclass[aps,prl,reprint,amsmath,nofootinbib]{revtex4-1}
\usepackage{hyperref}
\usepackage{graphicx}
\usepackage{subfig}
\graphicspath{{fig/}}


\begin{document}

\title{Entropy production from the generalized mean of nuclear density}

\author{J.\ Scott Moreland}
\author{Jonah E.\ Bernhard}
\author{Steffen A.\ Bass}
\affiliation{Duke University}

\date{\today}


\begin{abstract}
   We investigate entropy deposition at mid-rapidity in pp, pA and AA systems using a generalized mean of projectile and target 
densities $M_p(T_A,T_B)$ which describes a family of vector norms spanning a continuum of projectile/target saturation scenarios. 
Results are presented for three well known subcases of the generalized mean, the arithmetic, geometric and harmonic forms and 
compared against charged particle production in a wide range of collision systems. When combined with partonic density fluctuations of 
the nucleons within each nucleus, the data is well described by a generalized mean with power $p$ close to zero. We encapsulate 
these results in a new model for relativistic hydrodynamic initial conditions.
\end{abstract}

\maketitle

\section{Intoduction}

The discovery of the quark-gluon plasma (QGP) at RHIC initiated a large scale effort to quantify the transport properties of 
hot and dense nuclear matter. The success of ideal hydrodynamics in describing the collectivity observed in ultra-relativistic 
heavy-ion collisions supports the production of a fluid-like QGP with minimal specific shear viscosity close to the proposed 
K.S.S. bound $\eta/s \ge 1/4\pi$ \cite{KSS}. 

Extractions of $\eta/s$ from experiment using viscous hydrodynamics describe the emergence of collective flow from spatial 
deformations of the produced fireball with a conversion efficiency proportional to the system's average specific shear viscosity [?]. 
Consequently, model-to-data predictions of $\eta/s$ are beholden to the veractity of theoretical models used to calculate entropy and 
energy deposition in the initial state.

\section{Motivation}

We consider an ultra-relativistic collision between a projectile nucleus with atomic mass number $A$ and target nucleus with atomic mass number $B$ characterized 
by collision impact parameter ${\bf b}$ in the transverse plane. Let $T_A$ and $T_B$ denote the transverse density of participant matter, i.e.\ partons, 
in each nucleus as measured by the collapse of projectile and target wavefunctions. We consider the existence of an eikonal mapping $f(T_A,T_B)$ that 
characterizes entropy deposition at mid-rapidity as a function of $T_A$ and $T_B$,
\begin{equation}
  \label{ansatz}
  \frac{dS({\bf r}_\perp)}{d^2r_\perp dy} \biggr \rvert_{y=0}  = f(T_A({\bf r}_\perp + {\bf b}/2),\,T_B({\bf r}_\perp - {\bf b}/2)).
\end{equation}

It's reasonable to insist that $f$ respect several key properties, namely that $f$ is smooth, monatonic and symmetric in 
the variables $T_A$ and $T_B$ and that it saturates in the limit $T_A \ll T_B$ and $T_B \ll T_A$.

Motivated by these stipulations, we consider the family of vector p-norms known as the generalized mean as a suitable ansatz for the mapping $f$,
\begin{equation}
 \label{generalized mean ansatz}
 f(T_A, T_B) \propto M_p(T_A,T_B) = \left( \frac{T_A^p + T_B^p}{2} \right)^{1/p}.
\end{equation}

For $p = \{-\infty,-1,0,1,\infty\}$, the generalized mean becomes minimum, harmonic, geometric, arithmetic and maximum functions respectively which can be shown
to obey the following ordering inequality, 
\begin{equation}
 M_{-\infty} \le M_{-1} \le M_{0} \le M_{1} \le M_{\infty}.
\end{equation}
In this sense, the continuous parameter $p$ characterizes the dependence of entropy deposition on the projectile-target thickness asymmetry, or equivalently the 
degree of collision saturation. This saturation on $p$ is best visualized by fixing the thickness of the projectile $T_A$ and varying the thickness of the target 
$T_B$ for different values of $p$ as shown in figure \ref{saturation}.
\begin{figure}[b]
 \includegraphics[width=\linewidth]{saturation.pdf}
 \caption{\label{saturation} Saturation in the generalized mean $M_p(1,T_B)$ for special forms $p=\{-1, 0, 1\}$.}
\end{figure}

It's illuminating to compare this ansatz against the commonly used wounded nucleon and binary collision parameterization where the transverse entropy (or energy) density 
is set proportional to a linear combination of the local density of participant and pair-wise collisions,
\begin{equation}
 \frac{dS({\bf r}_\perp)}{d^2r_\perp dy} \biggr \rvert_{y=0}  \propto \frac{1-\alpha}{2}\mbox{WN}({\bf r_\perp}) + \alpha\, \mbox{BC}({\bf r_\perp}) 
\end{equation}
where the wounded nucleon and binary collision densities scale with the participant thicknesses $T_A$ and $T_B$ as,
\begin{eqnarray}
 \cr \mbox{WN}(T_A,T_B) &\propto& T_A + T_B \\
 \label{binary collisions}
 \mbox{BC}(T_A,T_B) &\propto& T_A\,T_B.
\end{eqnarray}

It's worth noting that the $p=1$ generalized mean is simply a wounded nucleon model. This is however, where the simularities between the generalized
mean and the two-component ansatz end. While the generalized mean is strictly linear in symmetric collisions $T_A=T_B$, binary collision contributions to the 
two-component model introduce terms that are quadratic in $T$.

An interesting consequence of this non-linearity becomes manifest in central U-U collisions. The Uranium nucleus is highly deformed and prolonged along a preferred 
axis. This geometry allows for two distinct configurations of Uranium collisions in which the density of nuclear matter is fully overlapping - tip-to-tip collisions 
where the Uranium symmetry axis is alligned with the beam axis and body-on-body collisions where the Uranium symmetry axis is perpendicular to the beam axis. 

Tip-to-tip configurations generate more binary collisions than body-on-body configurations and consequently, binary collision scaling would predict that these 
collisions produce more charged particles. This multiplicity ordering of tip-to-tip and body-on-body configurations creates a surplus of highly elliptic 
body-on-body configurations just below the most central event class which is predominantly tip-to-tip. It was shown by the authors in [?] that this sorting predicts 
a curious knee shaped structure in the centrality dependent elliptic flow data. 

Recent $v_2\{2\}$ and $v_2\{4\}$ flow results from STAR found no such evidence of a knee-shaped structure in ultra-central U-U collisions \cite{UU-STAR}. It has been
suggested by the authors in [?] suggested that the existence of hot spots in the produced fireball may wash out the structure of the knee, the analysis model with a higher degree of fluctuations,

This suggests that non-linearities in the symmetric component of equation (\ref{ansatz}) are small and thus the symmetric scaling $M_p(T,T) \propto T$ employed by the generalized mean
is a reasonable approximation.

\cite{constraining-ic}

\section{The Model}

For simplicity, we refer to the generalized mean of projectile and target thicknesses hereafter as the \emph{reduced thickness} of projectile nucleus $A$ and 
target nucleus $B$,
\begin{equation}
 \label{reduced}
 T_R(p;x,y) \equiv \left( \frac{T_A({\bf x}+{\bf b}/2,y)^p + T_B({\bf x}-{\bf b}/2,y)^p}{2} \right)^{1/p}
\end{equation}
noting that the units of $T_R$ are also that of a nuclear thickness function. 

We employ a slightly modified definition of the participant thickness function which we denote $\tilde{T}$ in which each participant nucleon contributes with random 
weight $\gamma_i$ sampled from a two parameter Gamma distribution $\Gamma(\mu,k)$ with mean $\mu=1$ and shape parameter $k$. For a Monte-Carlo distribution of nucleon 
positions, this amounts to summing the fluctuated thickness functions of each participant nucleon $T_p(x,y)$,
\begin{equation}
 \label{fluctuated thickness}
 T(x,y) = \sum\limits_{i=1}^{N_{part}} \gamma_i\, T_p(x-x_i,y-y_i).
\end{equation}

These additional weights incorporate event-by-event fluctuations in the density of partons sampled in each nucleon-nucleon collision. The choice of
a Gamma distribution reflects a long history of fitting proton-proton multiplicity distributions using a Negative Binomial distribution which arrises naturally 
from a Gamma-Poisson mixture.

For the average transverse parton density $T_p(x,y)$ we use a Gaussian,
\begin{equation}
 T_p(x,y) = \frac{1}{\sqrt{2 \pi B}} \exp\left(-\frac{x^2+y^2}{2 B}\right) 
\end{equation}
although we stress that this is a rather crude description of the nucleon. Consequently, we restrict our attention to properties of the collision that are insensitive to details 
of the nucleonic substructure.

The transverse nucleon coordinates $(x_i,y_i)$ in equation (\ref{fluctuated thickness}) are sampled in the cms frame of the projectile and target nuclei for the systems of interest. We generate
these coordinates for lead nuclei using the realistic nucleon configurations calculated by Alvioli, Drescher and Strikman in reference \cite{nucleon-correlations} which incorporate short- 
and long-range nucleon-nucleon correlations guided by the findings of realistic calculations of one- and two-body densities in medium-heavy nuclei.

Participant nucleons are determined in the projectile and target nuclei using the probabilistic collision criteria described in reference \cite{proton-proton}. In this prescription, the 
probability for a pairwise collision of two nucleons separated by impact parameter $b$ is given by,
\begin{equation}
  P^{inel}_{NN}(b) = 1 - \exp \left(-\sigma_{gg} N_{gg}^2 T_{pp}(b) \right), \\[1ex]
\end{equation}
where $T_{pp}(b)$ denotes the proton-proton overlap function defined by,
\begin{equation}
 T_{pp}(b) = \int dx~dy \,f(x-b/2,y) \,f(x+b/2,y).
\end{equation}
The exponential pre-factor $\sigma_{gg} N_{gg}^2$ representing effective glue-glue cross-section and squared gluon count respectively, is tuned to reproduce the total 
inelastic proton-proton cross-section,
\begin{equation}
  \sigma^{inel}_{NN} = \int 2 \pi b \,db \, P_{NN}^{inel}(b).
\end{equation}
The pair-wise collision probability in equation (?) is then sampled independently for each pair of projectile and target nucleons to isolate those nucleons which 
participate in at least one collision. 

In this language, we fix the transverse density of deposited entropy according to
\begin{equation}
 \label{entropy deposition}
 \frac{dS({\bf r}_\perp)}{d^2r_\perp dy} \biggr \rvert_{y=0} = \zeta(\sqrt{s})\, T_R(p;{\bf r}_\perp)
\end{equation}
where $\zeta(\sqrt{s})$ is an undetermined energy-dependent normalization factor and $T_R(x,y)$ the reduced thickness function. The entire procedure thus amounts to calculating the participant 
thickness functions $T_A$ and $T_B$ in combination with expressions (\ref{reduced}) and (\ref{entropy deposition}) with only one additional caveat.



\section{Results}

\begin{figure*}[h!]
    \centering
    \subfloat[label 1]{\includegraphics[width=0.333\textwidth]{pp.pdf}}
    \subfloat[label 2]{\includegraphics[width=0.333\textwidth]{pPb.pdf}}
    \subfloat[label 3]{\includegraphics[width=0.333\textwidth]{PbPb.pdf}}
    \caption{2 Figures side by side}
    \label{fig:example}
\end{figure*}


\section{Summary}

\bibliography{sources}

\end{document}
